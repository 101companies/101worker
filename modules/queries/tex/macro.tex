% THANKS DUE TO http://tex.stackexchange.com/questions/4889/input-only-part-of-a-file
\makeatletter
\newread\pin@file
\newcounter{pinlineno}
\newcommand\pin@accu{}
\newcommand\pin@ext{pintmp}
% inputs #3, selecting only lines #1 to #2 (inclusive)
\newcommand*\partialinput [3] {%
  \IfFileExists{#3}{%
    \openin\pin@file #3%
    % skip lines 1 to #1 (exclusive)
    \setcounter{pinlineno}{1}%
    \@whilenum\value{pinlineno}<#1 \do{%
      \read\pin@file to\pin@line%
      \stepcounter{pinlineno}%
    }%
    % prepare reading lines #1 to #2 inclusive
    \addtocounter{pinlineno}{-1}%
    \let\pin@accu\empty%
    \begingroup%
    \endlinechar\newlinechar%
    \@whilenum\value{pinlineno}<#2 \do{%
      % use safe catcodes provided by e-TeX's \readline
      \readline\pin@file to\pin@line%
      \edef\pin@accu{\pin@accu\pin@line}%
      \stepcounter{pinlineno}%
    }%
    \closein\pin@file
    \expandafter\endgroup
    \scantokens\expandafter{\pin@accu}%
  }{%
    \errmessage{File `#3' doesn't exist!}%
  }%
}
\makeatother

\newcommand{\uri}[1]{\textsf{#1}}
\newcommand{\ooo}[1]{\textsf{101#1}}%
\newcommand{\todo}[2]{\noindent{}TODO (#1): #2}
\newcommand{\solasote}{{\itshape\textsf{SoLaSoTe}}}
\newcommand{\sparqlnote}[1]{
\paragraph{A note on SPARQL} #1
}

\definecolor{codebackground}{rgb}{0.97,0.97,0.97}
\definecolor{gray}{rgb}{0.5,0.5,0.5}
\definecolor{keyword}{rgb}{0.5,0.0,0.5}
\definecolor{comment}{rgb}{0.3,0.5,0.3}
\definecolor{code}{rgb}{0.0,0.0,0.3}
\definecolor{ncode}{rgb}{0.0,0.3,0.3}
\definecolor{scode}{rgb}{0.3,0.0,0.3}
\definecolor{static}{rgb}{0.0,0.0,0.3}

\lstset{%
numbers=none,
basicstyle=\sffamily\color{ncode}\small\itshape,%
%frame=single,
frame=none,
columns=fullflexible,
sensitive,%
%  fontadjust=true,%
  showstringspaces=false,%
  keywordstyle=\color{keyword},%
  commentstyle=\color{comment}\normalfont\small,%
%  tabsize=2,
%  numbers=left,                   % where to put the line-numbers
%  numberstyle=\tiny\color{gray},  % the style that is used for the line-numbers
  backgroundcolor=\color{codebackground},
 %breaklines=true
  morestring=[b]"'%
}

\newcommand{\sparql}[1]{%
\medskip

\noindent
\begin{boxedminipage}{\hsize}
\hfill{}Query \underline{#1.sparql}
\lstinputlisting[language=SPARQL]{../sparql/#1.sparql}
\end{boxedminipage}
\medskip
}

\newcommand{\partialOutputTabular}[4]{{\small

\medskip

\noindent
\begin{boxedminipage}{\hsize}
\hfill{}Output of query \underline{#4.sparql} (first few rows)

\noindent
\begin{center}
\colorbox{codebackground}{
\begin{tabular}#1
#2\\\hline
\partialinput{1}{#3}{../output/#4.txt}
...
\end{tabular}}
\end{center}
\end{boxedminipage}
\medskip
}}
\newcommand{\completeOutputTabular}[3]{{\small

\medskip

\noindent
\begin{boxedminipage}{\hsize}
\hfill{}Output of query \underline{#3.sparql}

\noindent
\begin{center}
\colorbox{codebackground}{
\begin{tabular}#1
#2\\\hline
\input{../output/#3.txt}
\end{tabular}}
\end{center}
\end{boxedminipage}
\medskip
}}
\newcommand{\partialOutput}[2]{%

\medskip

\noindent
\begin{boxedminipage}{\hsize}
\hfill{}Output of query \underline{#2.sparql} (first few rows)
\lstinputlisting[language=CSV,lastline=#1]{../output/#2.txt}
{\large{}...}
\end{boxedminipage}
\medskip
}

\newcommand{\completeOutput}[1]{%

\medskip

\noindent
\begin{boxedminipage}{\hsize}
\hfill{}Output of query \underline{#1.sparql}
\lstinputlisting[language=CSV]{../output/#1.txt}
\end{boxedminipage}
\medskip
}



\newcommand*{\Objective}[1]{\emph{Objective:} #1}
\newcommand*{\Output}[1]{\emph{Output:} #1}

\newcommand*{\Result}[1]{

    \lstinputlisting{../output/#1.txt}
}