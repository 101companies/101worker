%%%%%%%%%%%%%%%%%%%%%%%%%%%%%%%%%%%%%%%%%%%%%

\section{The properties of \solasote}
\label{S:properties}

\solasote's individuals are not just classified (see
\S\ref{S:classify}), but they are characterized in several other ways
through semantic properties in the sense of Semantic Web-like
triples. The underlying set of predicates is listed here:

\completeOutputTabular{{l|l}}{\textbf{predicate} & \textbf{comment}}{predicates}
%|

\noindent
For what it's worth, the list of predicates can be retrieved from
\solasote's triplestore as follows:

\sparql{predicates}

\noindent
Before we discuss these predicates in more detail, we should also
identify their types in the sense of the assumed domain and range for
each predicate:

\completeOutputTabular{{l|l|l|l}}{\textbf{predicate} & \textbf{domain} & \textbf{range}}{domainsAndRanges}
%|

\noindent
Again, for completeness' sake, the table has been produced by the
following query:

\sparql{domainsAndRanges}

\noindent
Now let's discuss these properties one by one, while also providing
typical examples. We pick a particular order that fits convenience of
explanation.

%%%%%%%%%%%%%%%%%%%%%%%%%%%%%%%%%%%%%%%%%%%%%

\subsection{sameAs, similarTo, linksTo} 

This is a family of predicates all concerned with linking individuals
of \solasote{} with external web-based resources, e.g., pages on
Wikipedia or resources according to DBpedia. Their meaning and purpose
is closely related to \uri{owl:sameAs}~\cite{owl} and variations that
are discussed by the Semantic Web community~\cite{HalpinHT11}. The use
of \uri{onto:sameAs} expresses that the \solasote{} individual and the
external resource's URL refer to the same thing. For instanceL

\sparql{predicateSameAs1}

\noindent
That is, there is an \solasote{} individual for the `Java Language
Specification' (JLS) of entity type \uri{onto:Document}; we use the
\uri{onto:sameAs} predicate to associate it with Oracle's authorative
source for the JLS.

The use of \uri{onto:similarTo} expresses that the \solasote{}
individual and the external's URL refer to closely related but notably
not the same things. An unspecific link to an external resource is
enabled by \uri{onto:linksTo}. We query the links for an illustrative
individual, \uri{concept:Monad}:

\sparql{predicateSameAs2}

\completeOutputTabular{{l|l}}{\textbf{predicate} & \textbf{url}}{predicateSameAs2}
%|

\noindent
That is, two pages, one on the Haskell wiki, another on Wikipedia, are
linked to in `sameAs' properties. The idea is here that these two
pages describe the notion of monad exactly in the functional
programming-centric (perhaps even Haskell-biased) way as intended for
\solasote. Another page on Wikipedia is linked to in a `similarTo'
property because it is concerned with the related notion of monad in
category theory. Finally, a page on Wikibooks is linked to in a
`linksTo' property to express that this page is not considered a
definitional resource of the notion at hand, but it does provide
(pedagogically) valuable information.

%%%%%%%%%%%%%%%%%%%%%%%%%%%%%%%%%%%%%%%%%%%%%

\subsection{uses} 
\label{S:uses}

Systems (i.e., technologies and \ooo{}'s contributions) can make use
of instruments (i.e., software languages, technologies, and
concepts). The property of a system to use an instrument expresses
that said instrument is used in the design or implementation or
execution of said system. This may be more or less observable from the
outside; such a property expresses knowledge about `internals'.

More specifically, use of a language should be understood as `some
artifact of the system being written in said language'. For instance:

\sparql{predicateUses1}

\noindent
Here, \uri{onto:haskellStarter} is a simple Haskell-based contribution
to \ooo{}. Use of a technology should be understood as `the system
being developed or executed with the help of said technology'. Use of
a concept should be understood as `the concept being exercised or
taken dependence on in the design or implementation or execution of
the system'. For instance:

\sparql{predicateUses2}

\noindent
Here, \url{tech:JAXB}\footnote{... \uri{onto:sameAs}
  \url{http://www.oracle.com/technetwork/articles/javase/index-140168.html}}
is the Java platform's technology for XML-data binding, which indeed
uses `Java annotations' for controling the mapping.

%%%%%%%%%%%%%%%%%%%%%%%%%%%%%%%%%%%%%%%%%%%%%

\subsection{supports}

A technology can support another technology in that it provides some
sort of interface in generalized sense (e.g., an I/O behavior or a
plug-in model) so that the supporting technology can be used with the
supported technology. For instance:

\sparql{predicateSupports1}

\noindent
Here, \uri{tech:CMake}\footnote{... \uri{onto:sameAs}
  \url{http://www.cmake.org/}} is a cross-platform, open-source build
system which supports \uri{tech:Make} in that it CMake can generate
native makefiles.

The predicate \uri{onto:supports} generalizes the situation of
technologies supporting technologies so that instruments (i.e.,
languages, technologies, and concepts) support other instruments. Here
are the additional situations:
%
\begin{itemize}
\item A technology supporting a language: said interface can be
  leveraged by using the language.
\item A technology supporting a concept: said interface (the use
  thereof) conforms to the concept.
\item A language supporting a concept: the language's characteristics
  support the concept.
\item A concept supporting an instrument: the concept is expected here
  to denote a class of technologies and languages. Thus, this
  situation effectively reduces to one mentioned before.
\end{itemize}

\noindent
A few illustrative support relationships are queried here:

\sparql{predicateSupports2}

\partialOutputTabular{{l|l}}{\textbf{subject} & \textbf{object}}{10}{predicateSupports2}
%|

%%%%%%%%%%%%%%%%%%%%%%%%%%%%%%%%%%%%%%%%%%%%%

\subsection{illustrates}

As much as a system may use some instrument, a feature description or
any sort of document may be said to illustrate some instrument the
point being that a description may not be able to claim `use' of the
instrument, but it may very well stipulate or explain or motivate its
use. For instance:

\sparql{predicateIllustrates1}

\noindent
The listed handbook is claimed to illustrate the concept of functional
data structures. The `illustrates' predicate is specifically helpful
in communicating the purpose of software features of \ooo{}'s
imaginary software system. Here is a query that looks up concepts
illustrated by the features:

\sparql{predicateIllustrates2}

\partialOutputTabular{{l|l|l}}{\textbf{feature} & \textbf{concept} & \textbf{headline}}{7}{predicateIllustrates2}
%|

%%%%%%%%%%%%%%%%%%%%%%%%%%%%%%%%%%%%%%%%%%%%%

\subsection{partOf}

Whole-part relationships are used in many areas of modeling; they make
sense for \solasote, too. That is, some kinds of \solasote{}
individuals may be composites of other kinds of \solasote{}
individuals. For instance:

\sparql{predicatePartOf1}

\noindent
That is, the Java compiler,
\uri{tech:javac},\footnote{... \uri{onto:sameAs}
  \url{http://en.wikipedia.org/wiki/Javac}} is part of the Java
Development Kit, \uri{tech:JDK}.\footnote{... \uri{onto:sameAs}
  \url{http://en.wikipedia.org/wiki/Java_Development_Kit}}. Operationally,
by installing JDK on a machine, one also gets the executable for the
Java compiler. Here is another example exercising another entity type
for whole-part relationships:

\sparql{predicatePartOf2}

\noindent
That is, the XPath query language for XML,
\uri{lang:XPath},\footnote{... \uri{onto:sameAs}
  \url{http://www.w3.org/TR/xpath/}} is part of the XSLT
transformation language for XML
Development Kit, \uri{lang:XSLT}.\footnote{... \uri{onto:sameAs}
  \url{http://www.w3.org/TR/xslt}}. The `part of' relationship must not
be confused here with a `subset off' relationship. That is, by saying
XPath is part of XSLT, we refer to the fact that XPath expressions can
be used in certain operand position in an XSLT program. 

%%%%%%%%%%%%%%%%%%%%%%%%%%%%%%%%%%%%%%%%%%%%%

\subsection{implements}

Systems (i.e., technologies and \ooo{}'s contributions) can implement
descriptions (i.e., features or documents). The idea is that the
descriptions serve essentially as requirements. For instance:

\sparql{predicateImplements1}

\noindent
Here, \uri{onto:haskellStarter} is again the simple Haskell-based
contribution to \ooo{}, which was exercised already earlier on. The
triple states that the contribution implements \uri{feature:Total} (a
feature for totaling all salaries in a company of \ooo{}'s
system). Here is another example:

\sparql{predicateImplements2}

\noindent
Here, \uri{tech:javac} refers to the `standard' Java compiler, as part
of JDK, and \uri{doc:JLS} refers again to the Java Language
Specification, as noted earlier. Clearly, the `standard' is supposed
to implement the language `standard'.

%%%%%%%%%%%%%%%%%%%%%%%%%%%%%%%%%%%%%%%%%%%%%

\subsection{memberOf}

TBD

\begin{comment}
Members of a theme
Members of a vocabulary
Members of a course
\end{comment}

%%%%%%%%%%%%%%%%%%%%%%%%%%%%%%%%%%%%%%%%%%%%%

\subsection{moreComplexThan}

TBD

\begin{comment}
Aha, moreComplexThan!?
- A contribution being more complex than another contribution
- A feature being more complex than another feature
- Does this sound like something applicable to language subset relationship?
\end{comment}

%%%%%%%%%%%%%%%%%%%%%%%%%%%%%%%%%%%%%%%%%%%%%

\subsection{basedOn, varies}

TBD

\begin{comment}
Contributions based on other contributions
Contributions varying other contributions
\end{comment}

%%%%%%%%%%%%%%%%%%%%%%%%%%%%%%%%%%%%%%%%%%%%%

\subsection{dependsOn}

TBD

%%%%%%%%%%%%%%%%%%%%%%%%%%%%%%%%%%%%%%%%%%%%%

\subsection{implies}

TBD

%%%%%%%%%%%%%%%%%%%%%%%%%%%%%%%%%%%%%%%%%%%%%

\subsection{designedBy, developedBy, reviewedBy}

TBD

%%%%%%%%%%%%%%%%%%%%%%%%%%%%%%%%%%%%%%%%%%%%%

\subsection{mentions}

\sparql{predicateMentions}

\completeOutputTabular{{l|l}}{\textbf{object} & \textbf{headline}}{predicateMentions}
%|

TBD

%%%%%%%%%%%%%%%%%%%%%%%%%%%%%%%%%%%%%%%%%%%%%

\subsection{profile, carries}

Cover as miscellaneous.

%%%%%%%%%%%%%%%%%%%%%%%%%%%%%%%%%%%%%%%%%%%%%
