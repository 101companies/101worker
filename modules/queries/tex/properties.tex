%%%%%%%%%%%%%%%%%%%%%%%%%%%%%%%%%%%%%%%%%%%%%

\section{Properties of \solasote}
\label{S:properties}

\solatsote{} leverages several properties to characterize and relate
individuals, as developed in the present section.  We begin by
providing an overview of the properties. Next, we describe
subproperties that are often available to better constrain domains and
ranges to well-established scenarios. Eventually, we describe the
properties in more detail and illustrate them by means of
representative triples.

%%%%%%%%%%%%%%%%%%%%%%%%%%%%%%%%%%%%%%%%%%%%%

\subsection{Propertiess--Overview}

There are these overall properties:

\completeOutputTabular{{l|l}}{\textbf{property} & \textbf{comment}}{predicates}
%|

\noindent
For what it's worth, the list of predicates can be retrieved from
\solasote's triplestore as follows:

\sparql{predicates}

\noindent
Before we discuss these predicates in more detail, we should also
identify their types in the sense of the assumed domain and range for
each predicate:

\completeOutputTabular{{l|l|l|l}}{\textbf{predicate} & \textbf{domain} & \textbf{range}}{domainsAndRanges}
%|

\noindent
Again, for completeness' sake, the table has been produced by the
following query:

\sparql{domainsAndRanges}

\noindent
Now let's discuss these properties one by one, while also providing
typical examples. We pick a particular order that fits convenience of
explanation.

%%%%%%%%%%%%%%%%%%%%%%%%%%%%%%%%%%%%%%%%%%%%%

\subsection{sameAs, similarTo, linksTo} 
\label{S:sameAs}
\label{S:similarTo}
\label{S:linksTo}
 
This is a family of predicates all concerned with linking individuals
of \solasote{} with external web-based resources, e.g., pages on
Wikipedia or resources according to DBpedia. Their meaning and purpose
is closely related to \uri{owl:sameAs}~\cite{owl} and variations that
are discussed by the Semantic Web community~\cite{HalpinHT11}. The use
of \uri{onto:sameAs} expresses that the \solasote{} individual and the
external resource's URL refer to the same thing. For instanceL

\sparql{predicateSameAs1}

\noindent
That is, there is an \solasote{} individual for the `Java Language
Specification' (JLS) of entity type \uri{onto:Document}; we use the
\uri{onto:sameAs} predicate to associate it with Oracle's authorative
source for the JLS.

The use of \uri{onto:similarTo} expresses that the \solasote{}
individual and the external's URL refer to closely related but notably
not the same things. An unspecific link to an external resource is
enabled by \uri{onto:linksTo}. We query the links for an illustrative
individual, \uri{concept:Monad}:

\sparql{predicateSameAs2}

\completeOutputTabular{{l|l}}{\textbf{predicate} & \textbf{url}}{predicateSameAs2}
%|

\noindent
That is, two pages, one on the Haskell wiki, another on Wikipedia, are
linked to in `sameAs' properties. The idea is here that these two
pages describe the notion of monad exactly in the functional
programming-centric (perhaps even Haskell-biased) way as intended for
\solasote. Another page on Wikipedia is linked to in a `similarTo'
property because it is concerned with the related notion of monad in
category theory. Finally, a page on Wikibooks is linked to in a
`linksTo' property to express that this page is not considered a
definitional resource of the notion at hand, but it does provide
(pedagogically) valuable information.

%%%%%%%%%%%%%%%%%%%%%%%%%%%%%%%%%%%%%%%%%%%%%

\subsection{uses} 
\label{S:uses}

Systems (i.e., technologies and \ooo{}'s contributions) can make use
of instruments (i.e., software languages, technologies, and
concepts). The property of a system to use an instrument expresses
that said instrument is used in the design or implementation or
execution of said system. This may be more or less observable from the
outside; such a property expresses knowledge about `internals'.

More specifically, use of a language should be understood as `some
artifact of the system being written in said language'. For instance:

\sparql{predicateUses1}

\noindent
Here, \uri{onto:haskellStarter} is a simple Haskell-based contribution
to \ooo{}. Use of a technology should be understood as `the system
being developed or executed with the help of said technology'. Use of
a concept should be understood as `the concept being exercised or
taken dependence on in the design or implementation or execution of
the system'. For instance:

\sparql{predicateUses2}

\noindent
Here, \url{tech:JAXB}\footnote{... \uri{onto:sameAs}
  \url{http://www.oracle.com/technetwork/articles/javase/index-140168.html}}
is the Java platform's technology for XML-data binding, which indeed
uses `Java annotations' for controling the mapping.

%%%%%%%%%%%%%%%%%%%%%%%%%%%%%%%%%%%%%%%%%%%%%

\subsection{supports}
\label{S:supports}

A technology can support another technology in that it provides some
sort of interface in generalized sense (e.g., an I/O behavior or a
plug-in model) so that the supporting technology can be used with the
supported technology. For instance:

\sparql{predicateSupports1}

\noindent
Here, \uri{tech:CMake}\footnote{... \uri{onto:sameAs}
  \url{http://www.cmake.org/}} is a cross-platform, open-source build
system which supports \uri{tech:Make} in that it CMake can generate
native makefiles.

The predicate \uri{onto:supports} generalizes the situation of
technologies supporting technologies so that instruments (i.e.,
languages, technologies, and concepts) support other instruments. Here
are the additional situations:
%
\begin{itemize}
\item A technology supporting a language: said interface can be
  leveraged by using the language.
\item A technology supporting a concept: said interface (the use
  thereof) conforms to the concept.
\item A language supporting a concept: the language's characteristics
  support the concept.
\item A concept supporting an instrument: the concept is expected here
  to denote a class of technologies and languages. Thus, this
  situation effectively reduces to one mentioned before.
\end{itemize}

\noindent
A few illustrative support relationships are queried here:

\sparql{predicateSupports2}

\partialOutputTabular{{l|l}}{\textbf{subject} & \textbf{object}}{10}{predicateSupports2}
%|

%%%%%%%%%%%%%%%%%%%%%%%%%%%%%%%%%%%%%%%%%%%%%

\subsection{illustrates}
\label{S:illustrates}

As much as a system may use some instrument, a feature description or
any sort of document may be said to illustrate some instrument the
point being that a description may not be able to claim `use' of the
instrument, but it may very well stipulate or explain or motivate its
use. For instance:

\sparql{predicateIllustrates1}

\noindent
The listed handbook is claimed to illustrate the concept of functional
data structures. The `illustrates' predicate is specifically helpful
in communicating the purpose of software features of \ooo{}'s
imaginary software system. Here is a query that looks up concepts
illustrated by the features:

\sparql{predicateIllustrates2}

\partialOutputTabular{{l|l|l}}{\textbf{feature} & \textbf{concept} & \textbf{headline}}{7}{predicateIllustrates2}
%|

%%%%%%%%%%%%%%%%%%%%%%%%%%%%%%%%%%%%%%%%%%%%%

\subsection{partOf}
\label{S:partOf}

Whole-part relationships are used in many areas of modeling; they make
sense for \solasote, too. That is, some kinds of \solasote{}
individuals may be composites of other kinds of \solasote{}
individuals. For instance:

\sparql{predicatePartOf1}

\noindent
That is, the Java compiler,
\uri{tech:javac},\footnote{... \uri{onto:sameAs}
  \url{http://en.wikipedia.org/wiki/Javac}} is part of the Java
Development Kit, \uri{tech:JDK}.\footnote{... \uri{onto:sameAs}
  \url{http://en.wikipedia.org/wiki/Java_Development_Kit}}. Operationally,
by installing JDK on a machine, one also gets the executable for the
Java compiler. Here is another example exercising another entity type
for whole-part relationships:

\sparql{predicatePartOf2}

\noindent
That is, the XPath query language for XML,
\uri{lang:XPath},\footnote{... \uri{onto:sameAs}
  \url{http://www.w3.org/TR/xpath/}} is part of the XSLT
transformation language for XML
Development Kit, \uri{lang:XSLT}.\footnote{... \uri{onto:sameAs}
  \url{http://www.w3.org/TR/xslt}}. The `part of' relationship must not
be confused here with a `subset off' relationship. That is, by saying
XPath is part of XSLT, we refer to the fact that XPath expressions can
be used in certain operand position in an XSLT program. 

%%%%%%%%%%%%%%%%%%%%%%%%%%%%%%%%%%%%%%%%%%%%%

\subsection{implements}
\label{S:implements}

Systems (i.e., technologies and \ooo{}'s contributions) can implement
descriptions (i.e., features or documents). The idea is that the
descriptions serve essentially as requirements. For instance:

\sparql{predicateImplements1}

\noindent
Here, \uri{onto:haskellStarter} is again the simple Haskell-based
contribution to \ooo{}, which was exercised already earlier on. The
triple states that the contribution implements \uri{feature:Total} (a
feature for totaling all salaries in a company of \ooo{}'s
system). Here is another example:

\sparql{predicateImplements2}

\noindent
Here, \uri{tech:javac} refers to the `standard' Java compiler, as part
of JDK, and \uri{doc:JLS} refers again to the Java Language
Specification, as noted earlier. Clearly, the `standard' is supposed
to implement the language `standard'.

%%%%%%%%%%%%%%%%%%%%%%%%%%%%%%%%%%%%%%%%%%%%%

\subsection{memberOf}
\label{S:memberOf}

\solasote{} organizes individuals in containers. There are the
following use cases: 
%
\begin{itemize}
\item Vocabularies as containers with terms (typically concepts) as
  members.
\item Courses as containers with scripts (units such as lectures) as
  members.
\item Themes as containers with (\ooo{}'s) contributions as members.
\end{itemize}
%
These kinds of containment are illustrated in turn.

\sparql{memberOfVocabulary}

\partialOutputTabular{{l|l}}{\textbf{concept} & \textbf{headline}}{7}{memberOfVocabulary}
%|

\sparql{memberOfCourse}

\partialOutputTabular{l}{\textbf{script}}{7}{memberOfCourse}

%%%%%%%%%%%%%%%%%%%%%%%%%%%%%%%%%%%%%%%%%%%%%

\subsection{Collection with \solasote}
\label{S:collect}

\solasote's individuals may engage in different kinds of collections. (As evident from the queries to follow, such engagement is expressed through the `memberOf' predicate, which is also discussed again in \S\ref{S:memberOf}.) We mentioned vocabularies, themes, and courses as forms of collections in \S\ref{S:entities}. Collection complements classification (\S\ref{S:classify}).

%%%%%%%%%%%%%%%%%%%%%%%%%%%%%%%%%%%%%%%%%%%%%

\subsubsection{Collection of concepts as vocabularies}

We refer back to \S\ref{S:vocabulary} for the entity type of vocabularies. For instance, the `Haskell vocabulary' collects concepts that are essentially specific to the Haskell style of functional programming or the Haskell community.

\sparql{haskellVocabulary}

\partialOutputTabular{{l|l}}{\textbf{concept} & \textbf{headline}}{7}{haskellVocabulary}
%|

%%%%%%%%%%%%%%%%%%%%%%%%%%%%%%%%%%%%%%%%%%%%%

\subsubsection{Collection of contributions as themes}

We refer back to \S\ref{S:theme} for the entity type of themes. For instance, the `Starter' theme collects contributions that demonstrate some simple features across the board, i.e., for different software languages without relying on `advanced' software technologies or software concepts.

\sparql{starterTheme}

\completeOutputTabular{{l|l}}{\textbf{contribution} & \textbf{headline}}{starterTheme}
%|

%%%%%%%%%%%%%%%%%%%%%%%%%%%%%%%%%%%%%%%%%%%%%

\subsubsection{Collection of scripts as courses}

TBD

%%%%%%%%%%%%%%%%%%%%%%%%%%%%%%%%%%%%%%%%%%%%%

\subsection{moreComplexThan}
\label{S:moreComplexThan}

TBD

\begin{comment}
Aha, moreComplexThan!?
- A contribution being more complex than another contribution
- A feature being more complex than another feature
- Does this sound like something applicable to language subset relationship?
\end{comment}

%%%%%%%%%%%%%%%%%%%%%%%%%%%%%%%%%%%%%%%%%%%%%

\subsection{basedOn, varies}
\label{S:basedOn}
\label{S:varies}

TBD

\begin{comment}
Contributions based on other contributions
Contributions varying other contributions
\end{comment}

%%%%%%%%%%%%%%%%%%%%%%%%%%%%%%%%%%%%%%%%%%%%%

\subsection{dependsOn}
\label{S:dependsOn}

\solasote{}'s individuals may depend on each other in different
ways. Scripts, i.e., course units such as lectures or labs, may depend
on each other in the sense that one unit builds upon content of
another unit. Consider the following dependencies for a course on
functional programming:

\sparql{dependsOnScript}

\completeOutputTabular{{l|l}}{\textbf{earlier} & \textbf{later}}{dependsOnScript}
%|

\noindent
For instance, the somewhat advanced topic of `functors' (and friends)
depends on prior coverage of `type-class polymorphism'. Clearly, such
dependency relationships can be used to arrange the units in an actual
sequential order and it helps self-learners in processing the content
in a reasonable order. Here is indeed a query which orders the scripts
in such a way; we also list the number of scripts that are
prerequistes for each script:

\sparql{prerequisites}

\completeOutputTabular{{l|l}}{\textbf{script} & \textbf{count}}{prerequisites}
%|

\noindent
Another kind of dependency concerns features of \ooo{}'s system. That
is, one feature may `imply' (say, depend on) another feature. For
instance, a feature to compute the `depth' of department nesting implies
the feature for `hierachical companies', as flat companies would not
give rise to a meaningful notion of depth. We query all feature
dependencies as follows:

\sparql{dependsOnFeature}

\completeOutputTabular{{l|l}}{\textbf{feature} & \textbf{implied}}{dependsOnFeature}
%|

\noindent
Yet another kind of dependency concerns instruments to depend on other
instruments in the sense that one instrument cannot be `reasonably'
used without the other. For illustration, consider the following
dependencies for Ruby on Rails:

\sparql{dependsOnInstrument}

\completeOutputTabular{{l|l}}{\textbf{dependency} & \textbf{headline}}{dependsOnInstrument}
%|

\noindent
For instance, there is no `reasonable' way of using Rails other than
complying with the architectural pattern of MVC. Also, web development
with Rails assumes that all functionality etc.\ is coded in Rails;
likewise for the other dependencies.

%%%%%%%%%%%%%%%%%%%%%%%%%%%%%%%%%%%%%%%%%%%%%

\subsection{designedBy, developedBy, reviewedBy}
\label{S:designedBy}
\label{S:developedBy}
\label{S:reviewedBy}

These properties are concerned with persons and specifically
their involvement in the design, development, and reviewing of
different kinds of \solasote{}'s individuals. This is a list of
relevant scenarios:
%
\begin{itemize}
\item Persons may design languages, technologies, (\ooo{}'s) contributions, and
  courses.
\item Persons may develop technologies and contributions.
\item Persons may review contributions.
\end{itemize}
%
\solasote{} leverages the entity type \uri{onto:Contributor} for
persons concerned with (\ooo{}'s) contributions. Here is query for the
contributions developed by one of the present document's authors:

\sparql{developedBy}

\partialOutputTabular{{l|l}}{\textbf{contribution} & \textbf{headline}}{7}{developedBy}
%|

%%%%%%%%%%%%%%%%%%%%%%%%%%%%%%%%%%%%%%%%%%%%%

\subsection{mentions}
\label{S:mentions}

The documentation of \solasote{} on the \ooo{wiki} may use
`semantically weak' references to \solasote's individuals. We call
them `semantically weak' in that these references would not use any of
the specific predicate, but they are essentially plain
hyperlinks. These references are represented through `mentions'
properties in \solasote. 

For instance, the following query lists all individuals mentioned by a
simple Haskell-based contribution to \ooo{},
\uri{contrib:haskellStarter}:

\sparql{predicateMentions}

\completeOutputTabular{{l|l}}{\textbf{object} & \textbf{headline}}{predicateMentions}
%|

\noindent
A good number of concepts is mentioned because they are presumably
demonstrated (`used') by the contribution. The feature `Total' is
mentioned because the documentation discusses some details of this
particular feature; other features are implemented, but not discussed
explicitly. The language Haskell is mentioned for obvious reasons.

Ideally, all mentioned individuals should also be linked in a
semantically strong way, i.e., by using one of the predicates other
than `mention'. This can be regarded as a quality criterion for the
documention on \ooo{wiki}.

%%%%%%%%%%%%%%%%%%%%%%%%%%%%%%%%%%%%%%%%%%%%%

\subsection{carries}
\label{S:carries}

A simple tagging scheme is used for \solasote{} so that one can
associate `tags' with individuals. As of writing, the only noteworthy
example of a tag in use is `Stub', which is used to keep track of
contributions whose documentation is essentially missing or blatantly
incomplete. This idea is inspired by Wikipedia's stub notion.

%%%%%%%%%%%%%%%%%%%%%%%%%%%%%%%%%%%%%%%%%%%%%
