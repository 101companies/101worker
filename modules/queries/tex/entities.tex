%%%%%%%%%%%%%%%%%%%%%%%%%%%%%%%%%%%%%%%%%%%%%

\section{Individuals of \solasote}
\label{S:entities}

The kinds of individuals in the ontology give rise to \solasote's set
of entity types, which we will discuss first. Then, we query the
ontology to look at illustrative representatives of each entity type.

%%%%%%%%%%%%%%%%%%%%%%%%%%%%%%%%%%%%%%%%%%%%%

\subsection{Entity types}

There are these entity types:

\completeOutputTabular{{l|l}}{\textbf{type} & \textbf{comment}}{entityTypes}
%|

\noindent
For what it's worth, the list of entity types can be retrieved from
\solasote's triplestore as follows:

\sparql{entityTypes}

\noindent
That is, the types of individuals are organized as subclasses of a
base type \uri{onto:Entity}, but we do not include subclasses of the
\solasote-specific type \uri{onto:Classifier} because these are
non-root types used for classification. This is explained in more
detail in \S\ref{S:classify}.

%%%%%%%%%%%%%%%%%%%%%%%%%%%%%%%%%%%%%%%%%%%%%

\subsection{Type `Language'}

The following query retrieves all software language:

\sparql{languages}

\noindent
As there are many languages, we order them by `popularity'. Below, we
show only the most popular languages. By popularity we mean the
numbers of any sort of subjects referring to them through any sort of
predicate. In this manner, we see presumably more well-known, less
obscure individuals.

\partialOutputTabular{{l|l}}{\textbf{language} & \textbf{headline}}{7}{languages}
%|

%%%%%%%%%%%%%%%%%%%%%%%%%%%%%%%%%%%%%%%%%%%%%

\subsection{Type `Technology'}

We apply the same kind of query as before:

\sparql{technologies}

\partialOutputTabular{{l|l}}{\textbf{technology} & \textbf{headline}}{7}{technologies}
%|

%%%%%%%%%%%%%%%%%%%%%%%%%%%%%%%%%%%%%%%%%%%%%

\subsection{Type `Concept'}

We apply the same kind of query as before:

\sparql{concepts}

\partialOutputTabular{{l|l}}{\textbf{concept} & \textbf{headline}}{7}{concepts}
%|

%%%%%%%%%%%%%%%%%%%%%%%%%%%%%%%%%%%%%%%%%%%%%

\subsection{Type `Vocabulary'}
\label{S:vocabulary}

Concepts can be collected in \emph{vocabularies}. The collected
concepts are supposedly used in a certain context of programming or
development or by a certain community. Let's have a look at the
vocabularies:

\sparql{vocabularies}

\completeOutputTabular{{l|l}}{\textbf{vocabulary} & \textbf{headline}}{vocabularies}
%|

\noindent
We refer to \S\ref{S:collect} for a deeper discussion of vocabularies.

%%%%%%%%%%%%%%%%%%%%%%%%%%%%%%%%%%%%%%%%%%%%%

\subsection{Type `Contribution'}

\solasote{} relies on the chrestomathy \ooo{} for evidence in the form
of small systems that exercise languages, technologies, and
concepts. These systems are called \emph{contributions}. We sort them
by popularity again:

\sparql{contributions}

\partialOutputTabular{{l|l}}{\textbf{contribution} & \textbf{headline}}{12}{contributions}
%|

%%%%%%%%%%%%%%%%%%%%%%%%%%%%%%%%%%%%%%%%%%%%%

\subsection{Type `Contributor'}

TBD

%%%%%%%%%%%%%%%%%%%%%%%%%%%%%%%%%%%%%%%%%%%%%

\subsection{Type `Feature'}

Contributions implement features of \ooo{}'s imaginary human resources
management system (the \ooo{system}; see the `implements' predicate in
\S\ref{S:implements}. We sort the features by popularity again:

\sparql{features}

\noindent
All features (as of writing) are shown here to convey that \ooo{}'s
set of features is meant to be manageable. The features at the bottom
of the list are potentially obscure, experimental, or outdated.

\completeOutputTabular{{l|l}}{\textbf{feature} & \textbf{headline}}{features}
%|

%%%%%%%%%%%%%%%%%%%%%%%%%%%%%%%%%%%%%%%%%%%%%

\subsection{Type `Theme'}
\label{S:theme}

Contributions can be collected in \emph{themes}. The collected
contributions (systems) are of interest to a certain stakeholder,
perhaps to persons with a specific learning objective. Let's have a
look at the themes:

\sparql{themes}

\partialOutputTabular{{l|l}}{\textbf{theme} & \textbf{headline}}{10}{themes}
%|

\noindent
We refer to \S\ref{S:collect} for a deeper discussion of themes.

%%%%%%%%%%%%%%%%%%%%%%%%%%%%%%%%%%%%%%%%%%%%%

\subsection{Type `Script'}

TBD

%%%%%%%%%%%%%%%%%%%%%%%%%%%%%%%%%%%%%%%%%%%%%

\subsection{Type `Course'}

TBD

%%%%%%%%%%%%%%%%%%%%%%%%%%%%%%%%%%%%%%%%%%%%%

\subsection{Type `Document'}

TBD

%%%%%%%%%%%%%%%%%%%%%%%%%%%%%%%%%%%%%%%%%%%%%

\subsection{Type `Tag'}

TBD

%%%%%%%%%%%%%%%%%%%%%%%%%%%%%%%%%%%%%%%%%%%%%

