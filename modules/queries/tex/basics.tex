%%%%%%%%%%%%%%%%%%%%%%%%%%%%%%%%%%%%%%%%%%%%%

\section{Basics of \solasote}

\subsection{Entity types of \solasote}

The ontology covers individuals of the following types; we also refer
to these types as `entity types':

\completeOutputTabular{{l|l}}{\textbf{type} & \textbf{comment}}{entityTypes}
%|

\noindent
For what it's worth, the list of entity types can be retrieved from
\solasote's triplestore as follows:

\sparql{entityTypes}

\noindent
That is, the types of individuals are organized as subclasses of a
base type \uri{onto:Entity}, but we do not include subclasses of the
\solasote-specific type \uri{onto:Classifier} because these are types
used for classification. This is explained in more detail in
\S\ref{S:schema}.

Let's have a look at illustrative individuals for some of the entity types. To this end, we use queries to retrieve all software languages, technologies, and concepts. We order them by `popularity', i.e., numbers of any sort of subjects referring to them through any sort of predicate. In this manner, we see presumably more well-known, less obscure individuals. Thus:

\sparql{languages}

\sparql{technologies}

\sparql{concepts}

\noindent
Here are the first few individuals for each of the queries:

\partialOutputTabular{{l|l}}{\textbf{language} & \textbf{headline}}{7}{languages}
%|

\partialOutputTabular{{l|l}}{\textbf{technology} & \textbf{headline}}{7}{technologies}
%|

\partialOutputTabular{{l|l}}{\textbf{concept} & \textbf{headline}}{7}{concepts}
%|

\todo{Ralf}{What to do about names starting in `101'.}

\todo{Ralf}{Clearly, some of the listed names are not proper concept names.}

Likewise, other types of entities may be explored, e.g.:

\begin{itemize}
\item Features
\item Contributions
\item Contributors
\end{itemize}

%%%%%%%%%%%%%%%%%%%%%%%%%%%%%%%%%%%%%%%%%%%%%
