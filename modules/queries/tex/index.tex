\documentclass{article}

\usepackage{fancyvrb}
\usepackage{tikz}
\usepackage{boxedminipage}
\usepackage{listings}
\usepackage{lstsparql}
\usepackage{lstcsv}
\usepackage{comment}
\usepackage{xcolor}
\usepackage{hyperref}

\newcommand{\uri}[1]{\textsf{#1}}
\newcommand{\ooo}[1]{\textsf{101#1}}%
\newcommand{\todo}[2]{\noindent{}TODO (#1): #2}
\newcommand{\solasote}{{\itshape\textsf{SoLaSoTe}}}
\newcommand{\sparqlnote}[1]{
\paragraph{A note on SPARQL} #1
}

\definecolor{codebackground}{rgb}{0.97,0.97,0.97}
\definecolor{gray}{rgb}{0.5,0.5,0.5}
\definecolor{keyword}{rgb}{0.5,0.0,0.5}
\definecolor{comment}{rgb}{0.3,0.5,0.3}
\definecolor{code}{rgb}{0.0,0.0,0.3}
\definecolor{ncode}{rgb}{0.0,0.3,0.3}
\definecolor{scode}{rgb}{0.3,0.0,0.3}
\definecolor{static}{rgb}{0.0,0.0,0.3}

\lstset{%
numbers=none,
basicstyle=\sffamily\color{ncode}\small\itshape,%
%frame=single,
frame=none,
columns=fullflexible,
sensitive,%
%  fontadjust=true,%
  showstringspaces=false,%
  keywordstyle=\color{keyword},%
  commentstyle=\color{comment}\normalfont\small,%
%  tabsize=2,
%  numbers=left,                   % where to put the line-numbers
%  numberstyle=\tiny\color{gray},  % the style that is used for the line-numbers
  backgroundcolor=\color{codebackground},
 %breaklines=true
  morestring=[b]"'%
}

\newcommand{\sparql}[1]{%
\medskip

\noindent
\begin{boxedminipage}{\hsize}
\hfill{}Query \underline{#1.sparql}
\lstinputlisting[language=SPARQL]{../sparql/#1.sparql}
\end{boxedminipage}
\medskip
}

\newcommand{\partialOutput}[2]{%

\medskip

\noindent
\begin{boxedminipage}{\hsize}
\hfill{}Output of query \underline{#2.sparql} (first few rows)
\lstinputlisting[language=CSV,lastline=#1]{../output/#2.txt}
{\large{}...}
\end{boxedminipage}
\medskip
}

\newcommand{\completeOutput}[1]{%

\medskip

\noindent
\begin{boxedminipage}{\hsize}
\hfill{}Output of query \underline{#1.sparql}
\lstinputlisting[language=CSV]{../output/#1.txt}
\end{boxedminipage}
\medskip
}



\newcommand*{\Objective}[1]{\emph{Objective:} #1}
\newcommand*{\Output}[1]{\emph{Output:} #1}

\newcommand*{\Result}[1]{

    \lstinputlisting{../output/#1.txt}
}

\title{Exploring the \solasote{} ontology\\
(Draft as of \today)}

\author{Ralf L\"ammel \and Martin Leinberger \and Andrei Varanovich\\
Fachbereich Informatik, Universit\"at Koblenz-Landau, Germany}

\begin{document}

\maketitle

\begin{abstract}
\todo{Ralf}{Write abstract.}
\end{abstract}

\tableofcontents

%%%%%%%%%%%%%%%%%%%%%%%%%%%%%%%%%%%%%%%%%%%%%

\section{Format of this document}

\todo{Ralf}{Write this section.}

%%%%%%%%%%%%%%%%%%%%%%%%%%%%%%%%%%%%%%%%%%%%%

\section{Query/output series}

%%%%%%%%%%%%%%%%%%%%%%%%%%%%%%%%%%%%%%%%%%%%%

\subsection{Entities of \solasote}

\solasote{} covers software languages. The following SPARQL query
retrieves all languages of the ontology:

\sparql{languages}

\sparqlnote{\todo{Ralf}{Explain new SPARQL constructs.}}

\noindent
Here is the first few languages sorted alphabetically:

\partialOutput{7}{languages}

\noindent
\solasote{} covers software technologies, too. Here, is a SPARQL query
that retrieves all technologies of the ontology:

\sparql{technologies}

\noindent
Here is the first few technologies sorted alphabetically:

\partialOutput{7}{technologies}

\noindent
\solasote{} covers software concepts, too. Here, is a SPARQL query
that retrieves all concepts of the ontology:

\sparql{concepts}

\noindent
Here is the first few concepts sorted alphabetically:

\partialOutput{7}{concepts}

\todo{Ralf}{What to do about names starting in `101'.}

\todo{Ralf}{Clearly, some of the listed names are not proper concept names.}

Likewise, other types of entities may be explored, e.g.:

\begin{itemize}
\item Features
\item Contributions
\item Contributors
\end{itemize}

%%%%%%%%%%%%%%%%%%%%%%%%%%%%%%%%%%%%%%%%%%%%%

\subsection{Classification of \solasote's entities}

\solasote{} organizes classifiers for software languages,
technologies, and concepts. In fact, it uses a class hierarchy (in the
sense of rdfs:subClassOf) with instances at the leaves (in the sense
of rdf:type) for classification.

All classifiers are concepts, but not all concepts are
classifiers. Each classifier is associated with a concept of the same
name with a different prefix. We see this situation by looking at the
first few concept in alphabetical order as well as the first few
classifiers, again in alphabetical order and with associated concepts: 

\sparql{concepts}

\partialOutput{7}{concepts}

\sparql{classifiers}

\partialOutput{7}{classifiers}

\noindent
Let us see now how classifiers are instantiated. Here is how we ask
whether \uri{lang:Java} is a (software) language:

\sparql{askJavaOfTypeLanguage}

\noindent
Evaluation of this query literally returns `true':

\completeOutput{askJavaOfTypeLanguage}

\noindent
Instead, we may also ask for all types of \uri{lang:Java}:

\sparql{typesOfJava}

\noindent
\uri{lang:Java} has indeed a number of types:

\completeOutput{typesOfJava}

\noindent
These types can be explained in the following way:
%
\begin{description}
%
\item[\uri{onto:Language}] This type stems from the entity's namespace
  on the \ooo{wiki}. That is, `Java' belongs to \ooo{wiki}'s namespace
  `Language' and thus, it gets associated with the type \uri{onto:Language}
  by the mapping from \ooo{wiki} to \solasote
%
\item[\uri{onto:OO\_programming\_language}] This type stems from a
  semantic property stated on \ooo{wiki}'s page for \uri{lang:Java}. That
  is, the
  page\footnote{\url{http://101companies.org/wiki/Language:Java}}
  contains markup for a relevant semantic property as follows:
\begin{center}
\begin{BVerbatim}
[[instanceOf::OO programming language]]
\end{BVerbatim}
\end{center}
\noindent
In this manner, the wiki directly declares `Java' to be of type `OO
programming language'.
%
\item[\uri{onto:Instrument}, \uri{onto:Entity}] These are supertypes
  of \uri{onto:Language}, as declared explicitly by the general class
  hierarchy for \solasote. Thus, these types are listed because they
  are inferred by the \emph{rdfs:subClassOf} hierarchy of \solasote.
%
\item[\uri{onto:Language}, \uri{onto:Programming\_language}]
  These are supertypes of \uri{onto:OO\_programming\_language}, as
  declared by semantic `isA' properties on the \ooo{wiki}. For
  instance, the
  page for \uri{onto:OO\_programming\_language}\footnote{\url{http://101companies.org/wiki/OO_programming_language}}
  contains markup for a relevant semantic property as follows:
\begin{center}
\begin{BVerbatim}
[[isA::Programming language]]
\end{BVerbatim}
\end{center}
\noindent
Thus, these types are listed because they are inferred by the
\emph{rdfs:subClassOf} hierarchy of \solasote. The type
\uri{onto:Language} is a shorthand for what is called `software
language' on the wiki. The type \uri{onto:Programming\_language} is a
proper classifier of (software) languages (just as much as
\uri{onto:OO\_programming\_language}).
%
\item[\uri{onto:WikiPage}] This type conveys that the resource at hand
  corresponds to a wiki page on the \ooo{wiki}. This is basically
  true for all resources of type \uri{onto:Entity}. The type factors
  all more representational properties of entities; see
  \S\ref{S:mapping}.
%
%\item[\uri{rdfs:Resource}] This is the root type of all resources
%  based on RDFS' underlying type system.
\end{description}
%
We can also query the supertypes of a given type. For instance, let's
query the supertypes of \uri{onto:OO\_programming\_language}, as we
have encountered this type previously:

\todo{Martin}{Once the name mapping of 26 Aug is in place, then
  uri{rdfs:Resource} should be gone from the output; ditto \uri{onto:Software\_language}.}

\sparql{supertypesOfOoProgrammingLanguage}

\noindent
These are all the supertypes of \uri{onto:OO\_programming\_language};
please note that all the types but \uri{onto:Programming\_language}
are inferred and that inference clearly assumes reflexivity, since
\uri{onto:OO\_programming\_language} is returned as a supertype of itself:

\completeOutput{supertypesOfOoProgrammingLanguage}

\noindent
Alternatively, we can also query for all the instances of a certain
class. For instance, we can ask for all the OO programming languages:

\sparql{instancesOfOoProgrammingLanguage}

\noindent
These are some of the OO programming languages:

\partialOutput{7}{instancesOfOoProgrammingLanguage}

%%%%%%%%%%%%%%%%%%%%%%%%%%%%%%%%%%%%%%%%%%%%%

\subsection{Mapping from \ooo{wiki} to \solasote}
\label{S:mapping}

\todo{Ralf}{Write this section.}

\sparql{haskellHeadline}

\sparql{haskellWikiLink}

\completeOutput{haskellHeadline}

\completeOutput{haskellWikiLink}

%%%%%%%%%%%%%%%%%%%%%%%%%%%%%%%%%%%%%%%%%%%%%

\subsection{The schema of \solasote}

\todo{Ralf}{Write this section.}

\sparql{properties}

\todo{Martin and Andrei}{Integrate 101's properties into
  triplestore. This requires a mapping from the DSL to RDFS/Owl.}

\partialOutput{7}{properties}

%%%%%%%%%%%%%%%%%%%%%%%%%%%%%%%%%%%%%%%%%%%%%

\subsection{Popularity of \solasote's entities}

We may be interested in the popularity of different kinds of entities
in the ontology. For instance, we may use the number of distinct
contributions referring to an entity for ranking all entities of a
certain kind (such as languages, technologies, or concepts). 

\sparql{popularLanguages}

\sparqlnote{\todo{Ralf}{Explain new SPARQL constructs.}}

\partialOutput{7}{popularLanguages}

\sparql{popularTechnologies}

\partialOutput{7}{popularTechnologies}

\sparql{popularConcepts}

\partialOutput{7}{popularConcepts}

%%%%%%%%%%%%%%%%%%%%%%%%%%%%%%%%%%%%%%%%%%%%%

\subsection{Validation of \solasote}

\todo{Ralf}{Write this section.}

\todo{Martin}{Complete sandbox upload on 101worker.}

%%%%%%%%%%%%%%%%%%%%%%%%%%%%%%%%%%%%%%%%%%%%%

\subsection{Concluding remarks}

\todo{Ralf}{Write this section.}

%%%%%%%%%%%%%%%%%%%%%%%%%%%%%%%%%%%%%%%%%%%%%

\begin{comment}

\Objective{If we maintained a “status” property for each page (such as “needs work”),
then we could try to find developers who would be likely able to improve content on the grounds of
their contribution pages referring to the wiki pages. Some fairness would need to be applied to we try to
spread work over multiple people (subject to email invitation)}

\Output{A simple version: Associate a developer with a stub page based on the
number of references of that developer’s contribution pages to this page}

\Objective{Find the contribution that uses a given language and exercises a given
concept such that there is no other contribution with less features, languages, technologies, and concepts involved}

\Output{TBD}

\Objective{The query should identify the concepts that are uniquely connected to specific programming paradigms}

\Output{Some sort of table per paradigm listing the most frequently occurring concepts.
Arguably, we also need to see a table of concepts that are not uniquely connected with one paradigm,
as it is this table that we need to use for debugging the wiki (such as by adding ignore-mentioning markup}

\Result{conceptsToParadigm}


\Objective{The query should identify the paradigms of different developers}
output{A simple matrix showing developers in one dimension and paradigms in the other dimension.
In this manner, we also cater for multi-paradigm developers}

\Output{TBD}


\Objective{The query returns disjointness violations}

\Output{The list of instances that violate disjointness constraints}


\Objective{Determine technologies or languages that seem to be associated with one dominant knowledge holder}

\Output{Show technologies and languages that seem to have such a dominant knowledge holder.
Should this be a matrix or just a plain flat table? We need to discuss this some more.
Perhaps we could add some extra metrics to show, for example,
how severe this knowledge island is in terms of affected contributions or lack of any secondary knowledge holder}


\Objective{Given a language and feature, find the “simplest” contribution of that feature in that language}

\Output{Simple matrix for languages times features with links to the contributions in the cells}


\Objective{TBD}

\Output{TBD}


\Objective{TBD}

\Output{TBD}


\Objective{TBD}

\Output{TBD}


\Objective{Given a technology and feature, find the “simplest” contribution of that feature in that technology}

\Output{Simple matrix for technologies times features with links to the contributions in the cells}

\Objective{Show what languages and technologies are used together}

\Output{TBD}


\end{comment}

\end{document}