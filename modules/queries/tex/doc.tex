\documentclass{article}

\usepackage{fancyvrb}
\usepackage{tikz}
\usepackage{boxedminipage}
\usepackage{listings}
\usepackage{lstsparql}
\usepackage{lstcsv}
\usepackage{comment}
\usepackage{xcolor}
\usepackage{hyperref}
% THANKS DUE TO http://tex.stackexchange.com/questions/4889/input-only-part-of-a-file
\makeatletter
\newread\pin@file
\newcounter{pinlineno}
\newcommand\pin@accu{}
\newcommand\pin@ext{pintmp}
% inputs #3, selecting only lines #1 to #2 (inclusive)
\newcommand*\partialinput [3] {%
  \IfFileExists{#3}{%
    \openin\pin@file #3%
    % skip lines 1 to #1 (exclusive)
    \setcounter{pinlineno}{1}%
    \@whilenum\value{pinlineno}<#1 \do{%
      \read\pin@file to\pin@line%
      \stepcounter{pinlineno}%
    }%
    % prepare reading lines #1 to #2 inclusive
    \addtocounter{pinlineno}{-1}%
    \let\pin@accu\empty%
    \begingroup%
    \endlinechar\newlinechar%
    \@whilenum\value{pinlineno}<#2 \do{%
      % use safe catcodes provided by e-TeX's \readline
      \readline\pin@file to\pin@line%
      \edef\pin@accu{\pin@accu\pin@line}%
      \stepcounter{pinlineno}%
    }%
    \closein\pin@file
    \expandafter\endgroup
    \scantokens\expandafter{\pin@accu}%
  }{%
    \errmessage{File `#3' doesn't exist!}%
  }%
}
\makeatother

\newcommand{\uri}[1]{\textsf{#1}}
\newcommand{\ooo}[1]{\textsf{101#1}}%
\newcommand{\todo}[2]{\noindent{}TODO (#1): #2}
\newcommand{\solasote}{{\itshape\textsf{SoLaSoTe}}}
\newcommand{\sparqlnote}[1]{
\paragraph{A note on SPARQL} #1
}

\definecolor{codebackground}{rgb}{0.97,0.97,0.97}
\definecolor{gray}{rgb}{0.5,0.5,0.5}
\definecolor{keyword}{rgb}{0.5,0.0,0.5}
\definecolor{comment}{rgb}{0.3,0.5,0.3}
\definecolor{code}{rgb}{0.0,0.0,0.3}
\definecolor{ncode}{rgb}{0.0,0.3,0.3}
\definecolor{scode}{rgb}{0.3,0.0,0.3}
\definecolor{static}{rgb}{0.0,0.0,0.3}

\lstset{%
numbers=none,
basicstyle=\sffamily\color{ncode}\small\itshape,%
%frame=single,
frame=none,
columns=fullflexible,
sensitive,%
%  fontadjust=true,%
  showstringspaces=false,%
  keywordstyle=\color{keyword},%
  commentstyle=\color{comment}\normalfont\small,%
%  tabsize=2,
%  numbers=left,                   % where to put the line-numbers
%  numberstyle=\tiny\color{gray},  % the style that is used for the line-numbers
  backgroundcolor=\color{codebackground},
 %breaklines=true
  morestring=[b]"'%
}

\newcommand{\sparql}[1]{%
\medskip

\noindent
\begin{boxedminipage}{\hsize}
\hfill{}Query \underline{#1.sparql}
\lstinputlisting[language=SPARQL]{../sparql/#1.sparql}
\end{boxedminipage}
\medskip
}

\newcommand{\partialOutputTabular}[4]{

\medskip

\noindent
\begin{boxedminipage}{\hsize}
\hfill{}Output of query \underline{#4.sparql} (first few rows)

\noindent
\begin{center}
\colorbox{codebackground}{
\begin{tabular}#1
#2\\\hline
\partialinput{1}{#3}{../output/#4.txt}
...
\end{tabular}}
\end{center}
\end{boxedminipage}
\medskip
}
\newcommand{\completeOutputTabular}[3]{

\medskip

\noindent
\begin{boxedminipage}{\hsize}
\hfill{}Output of query \underline{#3.sparql}

\noindent
\begin{center}
\colorbox{codebackground}{
\begin{tabular}#1
#2\\\hline
\input{../output/#3.txt}
\end{tabular}}
\end{center}
\end{boxedminipage}
\medskip
}
\newcommand{\partialOutput}[2]{%

\medskip

\noindent
\begin{boxedminipage}{\hsize}
\hfill{}Output of query \underline{#2.sparql} (first few rows)
\lstinputlisting[language=CSV,lastline=#1]{../output/#2.txt}
{\large{}...}
\end{boxedminipage}
\medskip
}

\newcommand{\completeOutput}[1]{%

\medskip

\noindent
\begin{boxedminipage}{\hsize}
\hfill{}Output of query \underline{#1.sparql}
\lstinputlisting[language=CSV]{../output/#1.txt}
\end{boxedminipage}
\medskip
}



\newcommand*{\Objective}[1]{\emph{Objective:} #1}
\newcommand*{\Output}[1]{\emph{Output:} #1}

\newcommand*{\Result}[1]{

    \lstinputlisting{../output/#1.txt}
}

\title{The \solasote{} ontology\\
(Draft as of \today)}

\author{Ralf L\"ammel \and Martin Leinberger \and Andrei Varanovich\\
{\normalsize Fachbereich Informatik, Universit\"at Koblenz-Landau, Germany}}

\begin{document}

\maketitle

\begin{abstract}
  This document describes and exercises the \solasote{} ontology for
  software languages, software technologies, and software concepts, as
  they are relevant for programming and software engineering. Evidence
  from the software chrestomathy of the 101 project is used to
  demonstrate the ontology.
\end{abstract}

%%%%%%%%%%%%%%%%%%%%%%%%%%%%%%%%%%%%%%%%%%%%%

\tableofcontents

%%%%%%%%%%%%%%%%%%%%%%%%%%%%%%%%%%%%%%%%%%%%%

%%%%%%%%%%%%%%%%%%%%%%%%%%%%%%%%%%%%%%%%%%%%%

\section{Introduction}

\todo{Ralf}{Write this section.}

\subsubsection*{Road-map of the paper}

\begin{itemize}
\item \S\ref{S:prefixes} lists the prefixes of \solasote.
\item \S\ref{S:entities} gives a quick introduction to \solasote.
\item \S\ref{S:classify} describes classification as provided by \solasote.
\item \S\ref{S:properties} describes the properties of \solasote.
\item \S\ref{S:wiki} describes the wiki-to-triplestore mapping for \solasote.
\item \S\ref{S:explore} explores \solasote{} for usecases by means of queries.
\item \S\ref{S:validate} discusses the issue of validation for \solasote.
\item \S\ref{S:concl} concludes this document.
\end{itemize}

%%%%%%%%%%%%%%%%%%%%%%%%%%%%%%%%%%%%%%%%%%%%%

%%%%%%%%%%%%%%%%%%%%%%%%%%%%%%%%%%%%%%%%%%%%%

\section{Individuals of \solasote}
\label{S:individuals}

The kinds of individuals in the ontology give rise to what is called
\solasote's set of \emph{entity types}, which are developed in this
section. We begin by providing an overview of these types (say,
classes, in the sense of \emph{RDFS}). Next, we describe these types
in more detail and illustrate them by means of representative
individuals. Finally, we discuss classification, as applied to the
individuals, thereby establishing a taxnonomy-like hierarchy with the
entity types serving as the roots of classification.

%%%%%%%%%%%%%%%%%%%%%%%%%%%%%%%%%%%%%%%%%%%%%

\subsection{Entity types--Overview}

There are these entity types:

\completeOutputTabular{{l|l}}{\textbf{type} & \textbf{comment}}{entityTypes}
%|

\noindent
For what it's worth, the list of entity types can be retrieved from
\solasote's triplestore as follows:

\sparql{entityTypes}

\noindent
That is, the types of individuals are organized as subclasses of a
base type \uri{onto:Entity}, but we do not include subclasses of the
\solasote-specific type \uri{onto:Classifier} because these are
non-root types used for classification, as explained in detail in
\S\ref{S:classify}.

The following query determines the number of individuals for each
type:

\sparql{individuals}

\completeOutputTabular{{l|l}}{\textbf{type} & \textbf{count}}{individuals}
%|

\noindent
For clarity, we also list the prefixes used for the different
\solasote{} resources. In this manner, we also provide more initial
insight into the entity types and relationships to other ontologies or
namespaces.

{\small

\begin{verbatim}
PREFIX rdf:<http://www.w3.org/1999/02/22-rdf-syntax-ns#>
PREFIX rdfs:<http://www.w3.org/2000/01/rdf-schema#>
PREFIX owl:<http://www.w3.org/2002/07/owl#>
PREFIX xsd:<http://www.w3.org/2001/XMLSchema#>
PREFIX onto:<http://101companies.org/ontology#>
PREFIX res:<http://101companies.org/resources#>
PREFIX tech:<http://101companies.org/resources/Technology#>
PREFIX lang:<http://101companies.org/resources/Language#>
PREFIX concept:<http://101companies.org/resources/Concept#>
PREFIX voc:<http://101companies.org/resources/Vocabulary#>
PREFIX doc:<http://101companies.org/resources/Document#>
PREFIX feature:<http://101companies.org/resources/Feature#>
PREFIX contrib:<http://101companies.org/resources/Contribution#>
PREFIX theme:<http://101companies.org/resources/Theme#>
PREFIX contributor:<http://101companies.org/resources/Contributor#>
PREFIX course:<http://101companies.org/resources/Course#>
PREFIX script:<http://101companies.org/resources/Script#>
PREFIX tag:<http://101companies.org/resources/Tag#>
PREFIX sesame:<http://www.openrdf.org/schema/sesame#>
PREFIX foaf:<http://xmlns.com/foaf/0.1/>
\end{verbatim}

}

% PREFIX ns1:<http://www.w3.org/1999/02/22->

\noindent
The prefixes can be explained as follows:
%
\begin{description}
\item[rdf] The RDF data model.\footnote{\url{http://101companies.org/wiki/Language:RDF}}
\item[rdfs] The schema for RDF.\footnote{\url{http://101companies.org/wiki/Language:RDFS}}
\item[owl] The Web Ontology Language.\footnote{\url{http://101companies.org/wiki/Language:OWL}}
\item[xsd] XML Schema for data types.\footnote{\url{http://www.w3.org/2001/XMLSchema}}
\item[onto] Classes of the \solasote{} ontology.
\item[res] Individuals of \solasote.
\item[lang] Language individuals of \solasote.
\item[tech] Technology individuals of \solasote.
\item[concept] Concept individuals of \solasote.
\item[voc] Vocabularies as collections of \solasote{} concepts.
\item[doc] Document individuals of \solasote.
\item[feature] Features of 101 as individuals of \solasote.
\item[contrib] Contributions of 101 as individuals of \solasote.
\item[theme] Themes as collections of contributions.
\item[contributor] Contributors of 101 as individuals of \solasote.
\item[course] Courses making use of \solasote.
\item[script] Scripts (as parts of courses) making use of \solasote.
\item[tag] Tags applied to individuals of \solasote.
\item[sesame] The namespace of the Sesame framework.\footnote{\url{http://101companies.org/wiki/Technology:Sesame}}
\item[foaf] The ontology of the `Friend of a Friend' project.\footnote{\url{http://www.foaf-project.org/}}
\end{description}

%%%%%%%%%%%%%%%%%%%%%%%%%%%%%%%%%%%%%%%%%%%%%

\subsection{Entity types--Details}

%%%%%%%%%%%%%%%%%%%%%%%%%%%%%%%%%%%%%%%%%%%%%

\subsubsection{Type `Language'}

The following query retrieves all software languages:

\sparql{languages}

\noindent
As there are many languages, we order them by `popularity'. Below, we
show only the most popular languages. By popularity we mean the
numbers of any sort of subjects referring to the languages---through
any sort of predicate. In this manner, we see presumably more
well-known, less obscure individuals.

\partialOutputTabular{{l|l}}{\textbf{language} & \textbf{headline}}{7}{languages}
%|

%%%%%%%%%%%%%%%%%%%%%%%%%%%%%%%%%%%%%%%%%%%%%

\subsubsection{Type `Technology'}

We apply the same kind of query as before:

\sparql{technologies}

\partialOutputTabular{{l|l}}{\textbf{technology} & \textbf{headline}}{7}{technologies}
%|

%%%%%%%%%%%%%%%%%%%%%%%%%%%%%%%%%%%%%%%%%%%%%

\subsubsection{Type `Concept'}

We apply the same kind of query as before:

\sparql{concepts}

\partialOutputTabular{{l|l}}{\textbf{concept} & \textbf{headline}}{7}{concepts}
%|

%%%%%%%%%%%%%%%%%%%%%%%%%%%%%%%%%%%%%%%%%%%%%

\subsubsection{Type `Vocabulary'}
\label{S:vocabulary}

Concepts can be collected in \emph{vocabularies}. The collected
concepts are supposedly used in a certain context of programming or
development or by a certain community. There are not yet many
vocabularies; we can list them all:

\sparql{vocabularies}

\completeOutputTabular{{l|l}}{\textbf{vocabulary} & \textbf{headline}}{vocabularies}
%|

\noindent
Concepts are included into vocabularies by means of the `memberOf'
predicate; see \S\ref{S:collect}.

%%%%%%%%%%%%%%%%%%%%%%%%%%%%%%%%%%%%%%%%%%%%%

\subsubsection{Type `Contribution'}

\solasote{} relies on the chrestomathy \ooo{} for evidence in the form
of small systems that exercise languages, technologies, and
concepts. These systems are called \emph{contributions} (since someone
has to `contribute' them to the chrestomathy). We sort contributions by
popularity again:

\sparql{contributions}

\partialOutputTabular{{l|l}}{\textbf{contribution} & \textbf{headline}}{12}{contributions}
%|

%%%%%%%%%%%%%%%%%%%%%%%%%%%%%%%%%%%%%%%%%%%%%

\subsubsection{Type `Contributor'}

Contributions are designed, developed, and reviewed by
contributors. \ooo{} required GitHub identities for its
contributors. This also helps with identity management and
authentication. The most active (most referenced) contributors are
listed below:

\sparql{contributors}

\partialOutputTabular{{l}}{\textbf{contributor}}{7}{contributors}
%|

\noindent
The shown names can be directly used to look up these persons on
GitHub.\footnote{For instance, \emph{avaranovich} maps to
  \url{https://github.com/avaranovich}.}  Contributors are associated
with contributions by means of the `developedBy' predicate and
friends; see \S\ref{S:developedBy}.

%%%%%%%%%%%%%%%%%%%%%%%%%%%%%%%%%%%%%%%%%%%%%

\subsubsection{Type `Feature'}

Contributions implement features of \ooo{}'s imaginary human resources
management system (the \ooo{system}. We sort the features by
popularity again:

\sparql{features}

\noindent
All features (as of writing) are shown here to convey that \ooo{}'s
set of features is meant to be manageable. The features at the bottom
of the list are potentially obscure, experimental, or outdated.

\completeOutputTabular{{l|l}}{\textbf{feature} & \textbf{headline}}{features}
%|

\noindent
Contributions are associated with features by means of the
`implements' predicate; see \S\ref{S:implements}.

%%%%%%%%%%%%%%%%%%%%%%%%%%%%%%%%%%%%%%%%%%%%%

\subsubsection{Type `Theme'}
\label{S:theme}

Contributions can be collected in \emph{themes}. The assumption is
here that the collected contributions (systems) are of interest to a
certain stakeholder, perhaps to persons with a specific learning
objective. Let's have a look at the themes:

\sparql{themes}

\partialOutputTabular{{l|l}}{\textbf{theme} & \textbf{headline}}{10}{themes}
%|

\noindent
Contributions are included into themes by means of the `memberOf'
predicate; see \S\ref{S:collect}.

%%%%%%%%%%%%%%%%%%%%%%%%%%%%%%%%%%%%%%%%%%%%%

\subsubsection{Type `Script'}

Scripts are units of knowledge representation that are most likely
outlines for lectures or lab sessions. As of writing, they are used
exclusively indeed for lecture scripting in terms of the exercised
concepts, technologies, languages, and examples (contributions). Here
are some illustrations:

\sparql{scripts}

\partialOutputTabular{{l|l}}{\textbf{script} & \textbf{headline}}{7}{scripts}
%|

%%%%%%%%%%%%%%%%%%%%%%%%%%%%%%%%%%%%%%%%%%%%%

\subsubsection{Type `Course'}

Scripts can be collected in \emph{courses}. The collected scripts
(i.e., lectures or alike) are indeed meant to define the modules of an
actual course. At this point, there are only two courses that are
modeled in this way:

\sparql{courses}

\completeOutputTabular{{l|l}}{\textbf{course} & \textbf{headline}}{courses}
%|

\noindent
Scripts are included into courses by means of the `memberOf'
predicate; see \S\ref{S:collect}.

%%%%%%%%%%%%%%%%%%%%%%%%%%%%%%%%%%%%%%%%%%%%%

\subsubsection{Type `Document'}

Individuals (such as languages, technologies, and concepts) regularly
refer to `external' resources; see the `sameAs' predicate and friend
in \S\ref{S:sameAs} for details. In certain situations, it is
reasonable though to reify an external resource (a document in a broad
sense) as a \solasote{} individual. In this manner, it is possible to
make the external resource participate in all of \solasote{}'s
properties. We show the `headlines' of some of the documents reified
on \solasote:

\sparql{documents}

\partialOutputTabular{{l}}{\textbf{headline}}{3}{documents}
%|

\noindent
We only show the headlines (i.e., short explanations) of the
resources, not the names assigned to them in \solasote, as there is no
intuitive, comprehensive style for giving names to the
documents. There is also no comprehensive style for referring to them:
some documents may be referrable to through DOIs; others may have a
manifestation on Wikipedia; yet others may be best resolvable on
Amazon; etc. We refer to \S\ref{S:sameAs} for predicates that
associate \solasote{} individuals to external resources.

%%%%%%%%%%%%%%%%%%%%%%%%%%%%%%%%%%%%%%%%%%%%%

\subsubsection{Type `Tag'}

A simple tagging scheme is used for \solasote{} so that one can
associate `tags' with individuals. As of writing, only very few tags
are in use:

\sparql{tags}

\noindent
For instance, the `Stub' tag is used to keep track of contributions
whose documentation is essentially missing or blatantly
incomplete. This idea is inspired by Wikipedia's stub notion. We refer
to \S\ref{S:carries} for the `carries' predicate which is used to
associate individuals (such as contributions) with tags.

%%%%%%%%%%%%%%%%%%%%%%%%%%%%%%%%%%%%%%%%%%%%%

\subsection{Classification of individuals}
\label{S:classify}

\solasote's individuals are roughly classified on the grounds of the
entity types, as described previously. For instance, we may obviously
ask whether \uri{lang:Java} is a (software) language:

\sparql{javaOfTypeLanguage}

\noindent
Evaluation of this query literally returns `true':

\completeOutput{javaOfTypeLanguage}

\noindent
A more fine-grained classification is also supported on the grounds of a class hierarchy with the entity types as root types. These extra classes are called `classifiers'. For instance, \uri{lang:Java} is of the following (classifier) types:

\completeOutput{typesOfJava}

\noindent
The (classifier) types of an individual can be retrieved like this:

\sparql{typesOfJava}

\noindent
As the query clarifiers, classifier types can be explicitly selected by testing for the extra type \uri{onto:Classifier} of those classes, thereby not confusing them with general types of the ontology---such as \uri{onto:Language} or \uri{Entity}. The class(ifier) hierarchy can also be queried in itself---without starting from individuals. For instance, this is how we ask for the supertypes of the classifier

\sparql{supertypesOfOoProgrammingLanguage}

\completeOutput{supertypesOfOoProgrammingLanguage}

\noindent
This is how we query for all individuals with a certain classifier; in this case, we are interested in all OO programming languages:

\sparql{ooProgrammingLanguage}

\noindent
These are some OO programming languages:

\partialOutput{7}{ooProgrammingLanguage}

\solasote's approach towards classification makes one specific assumption.  For each classifier, there is a corresponding concept (\uri{onto:Concept}) of the same name modulo different prefixes (`onto' for classifieres, `concept' for concepts). Whether or not a concept has an associated classifier depends on the fact whether the concept is actually used for classification.

In \S\ref{S:entities}, we had queried for `popular' concepts regardless of whether they are (associated with) classifiers. Here is a similar query, which specifically focuses on `popular' classifiers:

\sparql{classifiers}

\noindent
In the query, we use a predicate `classifies' to look up the association between classifier and concept. Here are the first few classifiers returned by the query:

\partialOutputTabular{{l|l}}{\textbf{concept} & \textbf{headline}}{7}{classifiers}
%|

\noindent
Clearly, this list overlaps with the ranking of popular concepts overall; see again \S\ref{S:entities}. For comparison, here is query for `popular' non-classifier concepts:

\sparql{nonClassifiers}

\partialOutputTabular{{l|l}}{\textbf{concept} & \textbf{headline}}{7}{nonClassifiers}
%|

At this point, we have discussed \solasote's entity types (rooted in
\uri{onto:Entity} and \solasote's classification types forming a class
hierarchy rooted by the entity types with individuals as
instances. \solasote{} uses yet a few extra `base types' to capture
commonalities of entity types in certain contexts. Here is a list of
these types and the corresponding subClassOf relationships:

\completeOutputTabular{{l|l}}{\textbf{type} & \textbf{comment}}{baseTypes}
%|

\completeOutputTabular{{l|l}}{\textbf{subtype} & \textbf{supertype}}{baseTypeSubClassing}
%|

\noindent
For instance, both a software technology and a contribution (i.e., an
implementation of \ooo{}'s system) can be regarded as a `software
system'.  The idea is that these base types are convenient in setting
up \solasote's properties, as discussed in more detail in
\S\ref{S:properties}. For instance, both a technology and a
contribution may be said to be `developed by' a person. For
completeness' sake, the list of base types and the corresponding
subClassOf relationships can be retrieved from \solasote's triplestore
as follows:

\sparql{baseTypes}

\sparql{baseTypeSubClassing}

\noindent
Last but not least, \solasote{} also leverages external types, i.e.,
types of other ontologies. In fact, besides RDF and RDFS, \solasote{}
currently only uses these FOAF types:

\completeOutputTabular{{l|l}}{\textbf{subtype} & \textbf{supertype}}{externalTypes}
%|

\noindent
The following query results in the list shown above:

\sparql{externalTypes}

%%%%%%%%%%%%%%%%%%%%%%%%%%%%%%%%%%%%%%%%%%%%%

%%%%%%%%%%%%%%%%%%%%%%%%%%%%%%%%%%%%%%%%%%%%%

\section{The properties of \solasote}
\label{S:properties}

\solasote's individuals are not just classified (see
\S\ref{S:classify}), but they are characterized in several other ways
through semantic properties in the sense of Semantic Web-like
triples. The underlying set of predicates is listed here:

\completeOutputTabular{{l|l}}{\textbf{predicate} & \textbf{comment}}{predicates}
%|

\noindent
For what it's worth, the list of predicates can be retrieved from
\solasote's triplestore as follows:

\sparql{predicates}

\noindent
Before we discuss these predicates in more detail, we should also
identify their types in the sense of the assumed domain and range for
each predicate:

\completeOutputTabular{{l|l|l|l}}{\textbf{predicate} & \textbf{domain} & \textbf{range}}{domainsAndRanges}
%|

\noindent
Again, for completeness' sake, the table has been produced by the
following query:

\sparql{domainsAndRanges}

\noindent
Now let's discuss these properties one by one, while also providing
typical examples. We pick a particular order that fits convenience of
explanation.

%%%%%%%%%%%%%%%%%%%%%%%%%%%%%%%%%%%%%%%%%%%%%

\subsection{sameAs, similarTo, linksTo} 

This is a family of predicates all concerned with linking individuals
of \solasote{} with external web-based resources, e.g., pages on
Wikipedia or resources according to DBpedia. Their meaning and purpose
is closely related to \uri{owl:sameAs}~\cite{owl} and variations that
are discussed by the Semantic Web community~\cite{HalpinHT11}. The use
of \uri{onto:sameAs} expresses that the \solasote{} individual and the
external resource's URL refer to the same thing. For instanceL

\sparql{predicateSameAs1}

\noindent
That is, there is an \solasote{} individual for the `Java Language
Specification' (JLS) of entity type \uri{onto:Document}; we use the
\uri{onto:sameAs} predicate to associate it with Oracle's authorative
source for the JLS.

The use of \uri{onto:similarTo} expresses that the \solasote{}
individual and the external's URL refer to closely related but notably
not the same things. An unspecific link to an external resource is
enabled by \uri{onto:linksTo}. We query the links for an illustrative
individual, \uri{concept:Monad}:

\sparql{predicateSameAs2}

\completeOutputTabular{{l|l}}{\textbf{predicate} & \textbf{url}}{predicateSameAs2}
%|

\noindent
That is, two pages, one on the Haskell wiki, another on Wikipedia, are
linked to in `sameAs' properties. The idea is here that these two
pages describe the notion of monad exactly in the functional
programming-centric (perhaps even Haskell-biased) way as intended for
\solasote. Another page on Wikipedia is linked to in a `similarTo'
property because it is concerned with the related notion of monad in
category theory. Finally, a page on Wikibooks is linked to in a
`linksTo' property to express that this page is not considered a
definitional resource of the notion at hand, but it does provide
(pedagogically) valuable information.

%%%%%%%%%%%%%%%%%%%%%%%%%%%%%%%%%%%%%%%%%%%%%

\subsection{uses} 
\label{S:uses}

Systems (i.e., technologies and \ooo{}'s contributions) can make use
of instruments (i.e., software languages, technologies, and
concepts). The property of a system to use an instrument expresses
that said instrument is used in the design or implementation or
execution of said system. This may be more or less observable from the
outside; such a property expresses knowledge about `internals'.

More specifically, use of a language should be understood as `some
artifact of the system being written in said language'. For instance:

\sparql{predicateUses1}

\noindent
Here, \uri{onto:haskellStarter} is a simple Haskell-based contribution
to \ooo{}. Use of a technology should be understood as `the system
being developed or executed with the help of said technology'. Use of
a concept should be understood as `the concept being exercised or
taken dependence on in the design or implementation or execution of
the system'. For instance:

\sparql{predicateUses2}

\noindent
Here, \url{tech:JAXB}\footnote{... \uri{onto:sameAs}
  \url{http://www.oracle.com/technetwork/articles/javase/index-140168.html}}
is the Java platform's technology for XML-data binding, which indeed
uses `Java annotations' for controling the mapping.

%%%%%%%%%%%%%%%%%%%%%%%%%%%%%%%%%%%%%%%%%%%%%

\subsection{supports}

A technology can support another technology in that it provides some
sort of interface in generalized sense (e.g., an I/O behavior or a
plug-in model) so that the supporting technology can be used with the
supported technology. For instance:

\sparql{predicateSupports1}

\noindent
Here, \uri{tech:CMake}\footnote{... \uri{onto:sameAs}
  \url{http://www.cmake.org/}} is a cross-platform, open-source build
system which supports \uri{tech:Make} in that it CMake can generate
native makefiles.

The predicate \uri{onto:supports} generalizes the situation of
technologies supporting technologies so that instruments (i.e.,
languages, technologies, and concepts) support other instruments. Here
are the additional situations:
%
\begin{itemize}
\item A technology supporting a language: said interface can be
  leveraged by using the language.
\item A technology supporting a concept: said interface (the use
  thereof) conforms to the concept.
\item A language supporting a concept: the language's characteristics
  support the concept.
\item A concept supporting an instrument: the concept is expected here
  to denote a class of technologies and languages. Thus, this
  situation effectively reduces to one mentioned before.
\end{itemize}

\noindent
A few illustrative support relationships are queried here:

\sparql{predicateSupports2}

\partialOutputTabular{{l|l}}{\textbf{subject} & \textbf{object}}{10}{predicateSupports2}
%|

%%%%%%%%%%%%%%%%%%%%%%%%%%%%%%%%%%%%%%%%%%%%%

\subsection{illustrates}

As much as a system may use some instrument, a feature description or
any sort of document may be said to illustrate some instrument the
point being that a description may not be able to claim `use' of the
instrument, but it may very well stipulate or explain or motivate its
use. For instance:

\sparql{predicateIllustrates1}

\noindent
The listed handbook is claimed to illustrate the concept of functional
data structures. The `illustrates' predicate is specifically helpful
in communicating the purpose of software features of \ooo{}'s
imaginary software system. Here is a query that looks up concepts
illustrated by the features:

\sparql{predicateIllustrates2}

\partialOutputTabular{{l|l|l}}{\textbf{feature} & \textbf{concept} & \textbf{headline}}{7}{predicateIllustrates2}
%|

%%%%%%%%%%%%%%%%%%%%%%%%%%%%%%%%%%%%%%%%%%%%%

\subsection{partOf}

Whole-part relationships are used in many areas of modeling; they make
sense for \solasote, too. That is, some kinds of \solasote{}
individuals may be composites of other kinds of \solasote{}
individuals. For instance:

\sparql{predicatePartOf1}

\noindent
That is, the Java compiler,
\uri{tech:javac},\footnote{... \uri{onto:sameAs}
  \url{http://en.wikipedia.org/wiki/Javac}} is part of the Java
Development Kit, \uri{tech:JDK}.\footnote{... \uri{onto:sameAs}
  \url{http://en.wikipedia.org/wiki/Java_Development_Kit}}. Operationally,
by installing JDK on a machine, one also gets the executable for the
Java compiler. Here is another example exercising another entity type
for whole-part relationships:

\sparql{predicatePartOf2}

\noindent
That is, the XPath query language for XML,
\uri{lang:XPath},\footnote{... \uri{onto:sameAs}
  \url{http://www.w3.org/TR/xpath/}} is part of the XSLT
transformation language for XML
Development Kit, \uri{lang:XSLT}.\footnote{... \uri{onto:sameAs}
  \url{http://www.w3.org/TR/xslt}}. The `part of' relationship must not
be confused here with a `subset off' relationship. That is, by saying
XPath is part of XSLT, we refer to the fact that XPath expressions can
be used in certain operand position in an XSLT program. 

%%%%%%%%%%%%%%%%%%%%%%%%%%%%%%%%%%%%%%%%%%%%%

\subsection{implements}

Systems (i.e., technologies and \ooo{}'s contributions) can implement
descriptions (i.e., features or documents). The idea is that the
descriptions serve essentially as requirements. For instance:

\sparql{predicateImplements1}

\noindent
Here, \uri{onto:haskellStarter} is again the simple Haskell-based
contribution to \ooo{}, which was exercised already earlier on. The
triple states that the contribution implements \uri{feature:Total} (a
feature for totaling all salaries in a company of \ooo{}'s
system). Here is another example:

\sparql{predicateImplements2}

\noindent
Here, \uri{tech:javac} refers to the `standard' Java compiler, as part
of JDK, and \uri{doc:JLS} refers again to the Java Language
Specification, as noted earlier. Clearly, the `standard' is supposed
to implement the language `standard'.

%%%%%%%%%%%%%%%%%%%%%%%%%%%%%%%%%%%%%%%%%%%%%

\subsection{memberOf}

TBD

\begin{comment}
Members of a theme
Members of a vocabulary
Members of a course
\end{comment}

%%%%%%%%%%%%%%%%%%%%%%%%%%%%%%%%%%%%%%%%%%%%%

\subsection{moreComplexThan}

TBD

\begin{comment}
Aha, moreComplexThan!?
- A contribution being more complex than another contribution
- A feature being more complex than another feature
- Does this sound like something applicable to language subset relationship?
\end{comment}

%%%%%%%%%%%%%%%%%%%%%%%%%%%%%%%%%%%%%%%%%%%%%

\subsection{basedOn, varies}

TBD

\begin{comment}
Contributions based on other contributions
Contributions varying other contributions
\end{comment}

%%%%%%%%%%%%%%%%%%%%%%%%%%%%%%%%%%%%%%%%%%%%%

\subsection{dependsOn}

TBD

%%%%%%%%%%%%%%%%%%%%%%%%%%%%%%%%%%%%%%%%%%%%%

\subsection{implies}

TBD

%%%%%%%%%%%%%%%%%%%%%%%%%%%%%%%%%%%%%%%%%%%%%

\subsection{designedBy, developedBy, reviewedBy}

TBD

%%%%%%%%%%%%%%%%%%%%%%%%%%%%%%%%%%%%%%%%%%%%%

\subsection{mentions}

\sparql{predicateMentions}

\completeOutputTabular{{l|l}}{\textbf{object} & \textbf{headline}}{predicateMentions}
%|

TBD

%%%%%%%%%%%%%%%%%%%%%%%%%%%%%%%%%%%%%%%%%%%%%

\subsection{profile, carries}

Cover as miscellaneous.

%%%%%%%%%%%%%%%%%%%%%%%%%%%%%%%%%%%%%%%%%%%%%

%%%%%%%%%%%%%%%%%%%%%%%%%%%%%%%%%%%%%%%%%%%%%

\section{Mapping from \ooo{wiki} to \solasote}
\label{S:wiki}

\todo{Ralf}{Write this section.}

\begin{comment}
\sparql{haskellHeadline}

\sparql{haskellWikiLink}

\completeOutput{haskellHeadline}

\completeOutput{haskellWikiLink}
\end{comment}

%%%%%%%%%%%%%%%%%%%%%%%%%%%%%%%%%%%%%%%%%%%%%

\section{Exploration of \solasote}
\label{S:explore}

\todo{Ralf}{Write this section.}

%%%%%%%%%%%%%%%%%%%%%%%%%%%%%%%%%%%%%%%%%%%%%

\begin{comment}
\subsection{Popularity of \solasote's entities}

We may be interested in the popularity of different kinds of entities
in the ontology. For instance, we may use the number of distinct
contributions referring to an entity for ranking all entities of a
certain kind (such as languages, technologies, or concepts). 

%\sparql{popularLanguages}

%\sparqlnote{\todo{Ralf}{Explain new SPARQL constructs.}}

%\partialOutput{7}{popularLanguages}

%\sparql{popularTechnologies}

%\partialOutput{7}{popularTechnologies}

%\todo{Ralf}{Why is contrib:haskellEngineer a popular technology?}

%\sparql{popularConcepts}

%\partialOutput{7}{popularConcepts}
\end{comment}
%%%%%%%%%%%%%%%%%%%%%%%%%%%%%%%%%%%%%%%%%%%%%

\section{Validation of \solasote}
\label{S:validate}

\todo{Ralf}{Write this section.}

\todo{Martin}{Complete sandbox upload on 101worker.}

\sparql{uniqueType}

\sparql{testDomain}

\sparql{testRange}

%%%%%%%%%%%%%%%%%%%%%%%%%%%%%%%%%%%%%%%%%%%%%

\input{related}
%%%%%%%%%%%%%%%%%%%%%%%%%%%%%%%%%%%%%%%%%%%%%

\section{Concluding remarks}
\label{S:concl}

\todo{Ralf}{Write this section eventually.}

%%%%%%%%%%%%%%%%%%%%%%%%%%%%%%%%%%%%%%%%%%%%%


%%%%%%%%%%%%%%%%%%%%%%%%%%%%%%%%%%%%%%%%%%%%%

\bibliography{doc}
\bibliographystyle{abbrv}

%%%%%%%%%%%%%%%%%%%%%%%%%%%%%%%%%%%%%%%%%%%%%

\end{document}