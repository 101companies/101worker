\documentclass{llncs}

\usepackage{fancyvrb}
\usepackage{tikz}
\usepackage{boxedminipage}
\usepackage{listings}
\usepackage{lstsparql}
\usepackage{lstcsv}
\usepackage{comment}
\usepackage{xcolor}
\usepackage{hyperref}
% THANKS DUE TO http://tex.stackexchange.com/questions/4889/input-only-part-of-a-file
\makeatletter
\newread\pin@file
\newcounter{pinlineno}
\newcommand\pin@accu{}
\newcommand\pin@ext{pintmp}
% inputs #3, selecting only lines #1 to #2 (inclusive)
\newcommand*\partialinput [3] {%
  \IfFileExists{#3}{%
    \openin\pin@file #3%
    % skip lines 1 to #1 (exclusive)
    \setcounter{pinlineno}{1}%
    \@whilenum\value{pinlineno}<#1 \do{%
      \read\pin@file to\pin@line%
      \stepcounter{pinlineno}%
    }%
    % prepare reading lines #1 to #2 inclusive
    \addtocounter{pinlineno}{-1}%
    \let\pin@accu\empty%
    \begingroup%
    \endlinechar\newlinechar%
    \@whilenum\value{pinlineno}<#2 \do{%
      % use safe catcodes provided by e-TeX's \readline
      \readline\pin@file to\pin@line%
      \edef\pin@accu{\pin@accu\pin@line}%
      \stepcounter{pinlineno}%
    }%
    \closein\pin@file
    \expandafter\endgroup
    \scantokens\expandafter{\pin@accu}%
  }{%
    \errmessage{File `#3' doesn't exist!}%
  }%
}
\makeatother

\newcommand{\uri}[1]{\textsf{#1}}
\newcommand{\ooo}[1]{\textsf{101#1}}%
\newcommand{\todo}[2]{\noindent{}TODO (#1): #2}
\newcommand{\solasote}{{\itshape\textsf{SoLaSoTe}}}
\newcommand{\sparqlnote}[1]{
\paragraph{A note on SPARQL} #1
}

\definecolor{codebackground}{rgb}{0.97,0.97,0.97}
\definecolor{gray}{rgb}{0.5,0.5,0.5}
\definecolor{keyword}{rgb}{0.5,0.0,0.5}
\definecolor{comment}{rgb}{0.3,0.5,0.3}
\definecolor{code}{rgb}{0.0,0.0,0.3}
\definecolor{ncode}{rgb}{0.0,0.3,0.3}
\definecolor{scode}{rgb}{0.3,0.0,0.3}
\definecolor{static}{rgb}{0.0,0.0,0.3}

\lstset{%
numbers=none,
basicstyle=\sffamily\color{ncode}\small\itshape,%
%frame=single,
frame=none,
columns=fullflexible,
sensitive,%
%  fontadjust=true,%
  showstringspaces=false,%
  keywordstyle=\color{keyword},%
  commentstyle=\color{comment}\normalfont\small,%
%  tabsize=2,
%  numbers=left,                   % where to put the line-numbers
%  numberstyle=\tiny\color{gray},  % the style that is used for the line-numbers
  backgroundcolor=\color{codebackground},
 %breaklines=true
  morestring=[b]"'%
}

\newcommand{\sparql}[1]{%
\medskip

\noindent
\begin{boxedminipage}{\hsize}
\hfill{}Query \underline{#1.sparql}
\lstinputlisting[language=SPARQL]{../sparql/#1.sparql}
\end{boxedminipage}
\medskip
}

\newcommand{\partialOutputTabular}[4]{

\medskip

\noindent
\begin{boxedminipage}{\hsize}
\hfill{}Output of query \underline{#4.sparql} (first few rows)

\noindent
\begin{center}
\colorbox{codebackground}{
\begin{tabular}#1
#2\\\hline
\partialinput{1}{#3}{../output/#4.txt}
...
\end{tabular}}
\end{center}
\end{boxedminipage}
\medskip
}
\newcommand{\completeOutputTabular}[3]{

\medskip

\noindent
\begin{boxedminipage}{\hsize}
\hfill{}Output of query \underline{#3.sparql}

\noindent
\begin{center}
\colorbox{codebackground}{
\begin{tabular}#1
#2\\\hline
\input{../output/#3.txt}
\end{tabular}}
\end{center}
\end{boxedminipage}
\medskip
}
\newcommand{\partialOutput}[2]{%

\medskip

\noindent
\begin{boxedminipage}{\hsize}
\hfill{}Output of query \underline{#2.sparql} (first few rows)
\lstinputlisting[language=CSV,lastline=#1]{../output/#2.txt}
{\large{}...}
\end{boxedminipage}
\medskip
}

\newcommand{\completeOutput}[1]{%

\medskip

\noindent
\begin{boxedminipage}{\hsize}
\hfill{}Output of query \underline{#1.sparql}
\lstinputlisting[language=CSV]{../output/#1.txt}
\end{boxedminipage}
\medskip
}



\newcommand*{\Objective}[1]{\emph{Objective:} #1}
\newcommand*{\Output}[1]{\emph{Output:} #1}

\newcommand*{\Result}[1]{

    \lstinputlisting{../output/#1.txt}
}

\title{The \solasote{} ontology\\
(Draft as of \today)}

\author{Ralf L\"ammel \and Martin Leinberger \and Andrei Varanovich}

\institute{Fachbereich Informatik, Universit\"at Koblenz-Landau, Germany}

\begin{document}

\maketitle

\begin{abstract}
  This document describes and exercises the \solasote{} ontology for
  software languages, software technologies, and software concepts, as
  they are relevant for programming and software engineering. Evidence
  from the software chrestomathy of the 101 project is used to
  demonstrate the ontology.
\end{abstract}

%%%%%%%%%%%%%%%%%%%%%%%%%%%%%%%%%%%%%%%%%%%%%

%%%%%%%%%%%%%%%%%%%%%%%%%%%%%%%%%%%%%%%%%%%%%

\section{Introduction}

\todo{Ralf}{Write this section.}

\subsubsection*{Road-map of the paper}

\begin{itemize}
\item \S\ref{S:prefixes} lists the prefixes of \solasote.
\item \S\ref{S:entities} gives a quick introduction to \solasote.
\item \S\ref{S:classify} describes classification as provided by \solasote.
\item \S\ref{S:properties} describes the properties of \solasote.
\item \S\ref{S:wiki} describes the wiki-to-triplestore mapping for \solasote.
\item \S\ref{S:explore} explores \solasote{} for usecases by means of queries.
\item \S\ref{S:validate} discusses the issue of validation for \solasote.
\item \S\ref{S:concl} concludes this document.
\end{itemize}

%%%%%%%%%%%%%%%%%%%%%%%%%%%%%%%%%%%%%%%%%%%%%

%%%%%%%%%%%%%%%%%%%%%%%%%%%%%%%%%%%%%%%%%%%%%

\section{Individuals of \solasote}
\label{S:entities}

The ontology covers individuals of the following types; we also refer
to these types as `entity types':

\completeOutputTabular{{l|l}}{\textbf{type} & \textbf{comment}}{entityTypes}
%|

\noindent
For what it's worth, the list of entity types can be retrieved from
\solasote's triplestore as follows:

\sparql{entityTypes}

\noindent
That is, the types of individuals are organized as subclasses of a
base type \uri{onto:Entity}, but we do not include subclasses of the
\solasote-specific type \uri{onto:Classifier} because these are types
used for classification. This is explained in more detail in
\S\ref{S:schema}.

Let's have a look at illustrative individuals for some of the entity types. To this end, we use queries to retrieve all software languages, technologies, and concepts. We order them by `popularity', i.e., numbers of any sort of subjects referring to them through any sort of predicate. In this manner, we see presumably more well-known, less obscure individuals. Thus:

\sparql{languages}

\sparql{technologies}

\sparql{concepts}

\noindent
Here are the first few individuals for each of the queries:

\partialOutputTabular{{l|l}}{\textbf{language} & \textbf{headline}}{7}{languages}
%|

\partialOutputTabular{{l|l}}{\textbf{technology} & \textbf{headline}}{7}{technologies}
%|

\partialOutputTabular{{l|l}}{\textbf{concept} & \textbf{headline}}{7}{concepts}
%|

%%%%%%%%%%%%%%%%%%%%%%%%%%%%%%%%%%%%%%%%%%%%%

%%%%%%%%%%%%%%%%%%%%%%%%%%%%%%%%%%%%%%%%%%%%%

\section{Classification with \solasote}
\label{S:classify}

\solasote's individuals are roughly classified on the grounds of the entity types described previously. For instance, we may obviously ask whether \uri{lang:Java} is a (software) language:

\sparql{javaOfTypeLanguage}

\noindent
Evaluation of this query literally returns `true':

\completeOutput{javaOfTypeLanguage}

\noindent
A more fine-grained classification is also supported on the grounds of a class hierarchy with the entity types as root types. These extra classes are called `classifiers'. For instance, \uri{lang:Java} is of the following (classifier) types:

\completeOutput{typesOfJava}

\noindent
The (classifier) types of an individual can be retrieved like this:

\sparql{typesOfJava}

\noindent
As the query clarifiers, classifier types can be explicitly selected by testing for the extra type \uri{onto:Classifier} of those classes, thereby not confusing them with general types of the ontology---such as \uri{onto:Language} or \uri{Entity}. The class(ifier) hierarchy can also be queried in itself---without starting from individuals. For instance, this is how we ask for the supertypes of the classifier

\sparql{supertypesOfOoProgrammingLanguage}

\completeOutput{supertypesOfOoProgrammingLanguage}

\noindent
This is how we query for all individuals with a certain classifier; in this case, we are interested in all OO programming languages:

\sparql{instancesOfOoProgrammingLanguage}

\noindent
These are some OO programming languages:

\partialOutput{7}{instancesOfOoProgrammingLanguage}

\solasote's approach towards classification makes one specific assumption.  For each classifier, there is a corresponding concept (\uri{onto:Concept}) of the same name modulo different prefixes (`onto' for classifieres, `concept' for concepts). Whether or not a concept has an associated classifier depends on the fact whether the concept is actually used for classification.

In \S\ref{S:entities}, we had queried for `popular' concepts regardless of whether they are (associated with) classifiers. Here is a similar query, which specifically focuses on `popular' classifiers:

\sparql{classifiers}

\noindent
In the query, we use a predicate `classifies' to look up the association between classifier and concept. Here are the first few classifiers returned by the query:

\partialOutputTabular{{l|l}}{\textbf{concept} & \textbf{headline}}{7}{classifiers}
%|

\noindent
Clearly, this list overlaps with the ranking of popular concepts overall; see again \S\ref{S:entities}. For comparison, here is query for `popular' non-classifier concepts:

\sparql{nonClassifiers}

\partialOutputTabular{{l|l}}{\textbf{concept} & \textbf{headline}}{7}{nonClassifiers}
%|

\sparql{baseTypeSubClassing}

\completeOutputTabular{{l|l}}{\textbf{subtype} & \textbf{supertype}}{baseTypeSubClassing}
%|

\sparql{externalTypes}

\completeOutputTabular{{l|l}}{\textbf{subtype} & \textbf{supertype}}{externalTypes}
%|

%%%%%%%%%%%%%%%%%%%%%%%%%%%%%%%%%%%%%%%%%%%%%

%%%%%%%%%%%%%%%%%%%%%%%%%%%%%%%%%%%%%%%%%%%%%

\section{The schema of \solasote}
\label{S:schema}

\completeOutputTabular{{l|l|l|l}}{\textbf{property} & \textbf{comment}}{properties}
%|

\sparql{properties}

\completeOutputTabular{{l|l|l|l}}{\textbf{property} & \textbf{domain} & \textbf{range}}{propertyTypes}
%|

\sparql{properties}

%%%%%%%%%%%%%%%%%%%%%%%%%%%%%%%%%%%%%%%%%%%%%

%%%%%%%%%%%%%%%%%%%%%%%%%%%%%%%%%%%%%%%%%%%%%

\section{Mapping from \ooo{wiki} to \solasote}
\label{S:wiki}

\todo{Ralf}{Write this section.}

\begin{comment}
\sparql{haskellHeadline}

\sparql{haskellWikiLink}

\completeOutput{haskellHeadline}

\completeOutput{haskellWikiLink}
\end{comment}

%%%%%%%%%%%%%%%%%%%%%%%%%%%%%%%%%%%%%%%%%%%%%

\section{Exploration of \solasote}
\label{S:explore}

\todo{Ralf}{Write this section.}

%%%%%%%%%%%%%%%%%%%%%%%%%%%%%%%%%%%%%%%%%%%%%

\begin{comment}
\subsection{Popularity of \solasote's entities}

We may be interested in the popularity of different kinds of entities
in the ontology. For instance, we may use the number of distinct
contributions referring to an entity for ranking all entities of a
certain kind (such as languages, technologies, or concepts). 

%\sparql{popularLanguages}

%\sparqlnote{\todo{Ralf}{Explain new SPARQL constructs.}}

%\partialOutput{7}{popularLanguages}

%\sparql{popularTechnologies}

%\partialOutput{7}{popularTechnologies}

%\todo{Ralf}{Why is contrib:haskellEngineer a popular technology?}

%\sparql{popularConcepts}

%\partialOutput{7}{popularConcepts}
\end{comment}
%%%%%%%%%%%%%%%%%%%%%%%%%%%%%%%%%%%%%%%%%%%%%

\section{Validation of \solasote}
\label{S:validate}

\todo{Ralf}{Write this section.}

\todo{Martin}{Complete sandbox upload on 101worker.}

\sparql{uniqueType}

\sparql{testDomain}

\sparql{testRange}

%%%%%%%%%%%%%%%%%%%%%%%%%%%%%%%%%%%%%%%%%%%%%

%%%%%%%%%%%%%%%%%%%%%%%%%%%%%%%%%%%%%%%%%%%%%

\section{Concluding remarks}
\label{S:concl}

\todo{Ralf}{Write this section eventually.}

%%%%%%%%%%%%%%%%%%%%%%%%%%%%%%%%%%%%%%%%%%%%%


%%%%%%%%%%%%%%%%%%%%%%%%%%%%%%%%%%%%%%%%%%%%%

\end{document}