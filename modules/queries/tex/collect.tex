%%%%%%%%%%%%%%%%%%%%%%%%%%%%%%%%%%%%%%%%%%%%%

\section{Collection with \solasote}
\label{S:collect}

\solasote's individuals may engage in different kinds of collections. (As evident from the queries to follow, such engagement is expressed through the `memberOf' predicate, which is also discussed again in \S\ref{S:memberOf}.) We mentioned vocabularies, themes, and courses as forms of collections in \S\ref{S:entities}. Collection complements classification (\S\ref{S:classify}).

%%%%%%%%%%%%%%%%%%%%%%%%%%%%%%%%%%%%%%%%%%%%%

\subsection{Collection of concepts as vocabularies}

We refer back to \S\ref{S:vocabulary} for the entity type of vocabularies. For instance, the `Haskell vocabulary' collects concepts that are essentially specific to the Haskell style of functional programming or the Haskell community.

\sparql{haskellVocabulary}

\partialOutputTabular{{l|l}}{\textbf{concept} & \textbf{headline}}{7}{haskellVocabulary}
%|

%%%%%%%%%%%%%%%%%%%%%%%%%%%%%%%%%%%%%%%%%%%%%

\subsection{Collection of contributions as themes}

We refer back to \S\ref{S:theme} for the entity type of themes. For instance, the `Starter' theme collects contributions that demonstrate some simple features across the board, i.e., for different software languages without relying on `advanced' software technologies or software concepts.

\sparql{starterTheme}

\completeOutputTabular{{l|l}}{\textbf{contribution} & \textbf{headline}}{starterTheme}
%|

%%%%%%%%%%%%%%%%%%%%%%%%%%%%%%%%%%%%%%%%%%%%%

\subsection{Collection of scripts as courses}

TBD

%%%%%%%%%%%%%%%%%%%%%%%%%%%%%%%%%%%%%%%%%%%%%
