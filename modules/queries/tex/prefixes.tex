%%%%%%%%%%%%%%%%%%%%%%%%%%%%%%%%%%%%%%%%%%%%%

\section{Prefixes of \solasote}
\label{S:prefixes}

The ontology cares about these prefixes ((sub-)ontologies):

{\small

\begin{verbatim}
PREFIX rdf:<http://www.w3.org/1999/02/22-rdf-syntax-ns#>
PREFIX rdfs:<http://www.w3.org/2000/01/rdf-schema#>
PREFIX owl:<http://www.w3.org/2002/07/owl#>
PREFIX onto:<http://101companies.org/ontology#>
PREFIX res:<http://101companies.org/resources#>
PREFIX lang:<http://101companies.org/resources/Language#>
PREFIX tech:<http://101companies.org/resources/Technology#>
PREFIX concept:<http://101companies.org/resources/Concept#>
PREFIX voc:<http://101companies.org/resources/Vocabulary#>
PREFIX feature:<http://101companies.org/resources/Feature#>
PREFIX contrib:<http://101companies.org/resources/Contribution#>
PREFIX theme:<http://101companies.org/resources/Theme#>
PREFIX contributor:<http://101companies.org/resources/Contributor#>
PREFIX course:<http://101companies.org/resources/Course#>
PREFIX script:<http://101companies.org/resources/Script#>
PREFIX sesame:<http://www.openrdf.org/schema/sesame#>
PREFIX foaf ...
\end{verbatim}

}

% PREFIX ns1:<http://www.w3.org/1999/02/22->

\noindent
The prefixes can be explained as follows:
%
\begin{description}
\item[rdf] The RDF data model.\footnote{\url{http://101companies.org/wiki/Language:RDF}}
\item[rdfs] The schema for RDF.\footnote{\url{http://101companies.org/wiki/Language:RDFS}}
\item[owl] The Web Ontology Language.\footnote{\url{http://101companies.org/wiki/Language:OWL}}
\item[onto] Classes of the \solasote{} ontology.
\item[res] Individuals of \solasote.
\item[lang] Language individuals of \solasote.
\item[tech] Technology individuals of \solasote.
\item[concept] Concept individuals of \solasote.
\item[voc] Vocabularies organizing terms of \solasote.
\item[feature] Features of 101 as individuals of \solasote.
\item[contrib] Contributions of 101 as individuals of \solasote.
\item[theme] Collections of contributions.
\item[contributor] Contributors of 101 as individuals of \solasote.
\item[course] Courses making use of \solasote.
\item[script] Scripts (as parts of courses) making use of \solasote.
\item[sesame] The namespace of the Sesame framework.\footnote{\url{http://101companies.org/wiki/Technology:Sesame}}
\item[foaf] The ontology of the `Friend of a Friend' project.\footnote{\url{http://www.foaf-project.org/}}
\end{description}

%%%%%%%%%%%%%%%%%%%%%%%%%%%%%%%%%%%%%%%%%%%%%
